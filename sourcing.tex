\ifx\justbeingincluded\undefined
\input chappreamble.tex
\input foolthesis.tex
\fi

\chapter{Creation}

%Like all good things under and including the sun, Jun's hydroxyl radical experiment also begins with creation. Specifically, we create a beam by briefly allowing a high pressure gas into the vacuum chamber and allowing it to supersonically expand.

\section{Supersonic Expansion}

Discussion of relevant thermodynamics. Enthalpy conserved along flowlines. Energetics, etc.

\section{Pulsed Valves}

Pulsed valves solve what is otherwise a significant dilemma for a vacuum system, and enable significant improvements in brightness compared to their long-forgotten CW cousins.
The key challenge for a pulsed valve is to rapidly transition from leak-tight to fully formed, low impedance flow.
Several strategies exist for achieving this.

\subsection{Solenoid}

The classic solenoid valve is a good place to start, but suffers from heat dissipation challenges and induction limited rise-times.
The General Valve and the Jordan Valve are two examples used previously on Jun's OH experiment.

\subsection{Piezo-Transduction}

Utilizing piezoelectric materials is a competitive alternative to the solenoid, with no induction issues but somewhat choked flux.
A version of the valve taken from the Nesbitt group uses a drum-head mode piezoelectric transducer, with $120~\mu$s open time.
When I first joined the lab, some interest in running a cantilever piezo existed.
This would offer faster mechanical response, but reduced dry-force, limiting the range of supersonic expansion stagnation pressures.

\subsection{Miniature Solenoid}

Even-Lavie

\subsection{Lorentz Force}

The lorentz force exerted on a wire by a permanent magnet induced field containing the wire can also be exploited for operating valves.
The key advantage of this is the reduced obstruction of the required hardware for gas flow.
This style was designed in Nijmegen in the Netherlands, and a similar design was undertaken in our shop, though not yet completed or tested at the time of this writing.

\section{Discharge}

Photolysis of HNO3, electric discharge in H2O, H2O2. RF v DC. Conversion Efficiency.





\ifx\justbeingincluded\undefined
\bibliographystyle{unsrtDR}	% or "siam", or "alpha", etc.
\bibliography{allrefs}		% Bib database in "allrefs.bib"
\end{document}
\fi
