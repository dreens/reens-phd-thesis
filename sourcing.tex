\ifx\justbeingincluded\undefined
\input chappreamble.tex
\input foolthesis.tex
\fi

\chapter{Creation}

Like all things, Jun Ye's hydroxyl radical experiment also begins with creation; specifically, the creation of hydroxyl radicals via discharge during pulsed supersonic expansion. 
Such a supersonic expansion occurs when gas at pressure is allowed to escape into vacuum via a small orifice.
No better resource on the supersonic expansion or free jet seems to exist besides~\cite{Miller1988}, although a more modern compendium of results does also exist~\cite{Campargue2000}.
I will begin with supersonic expansions and the thermodynamics thereof, before moving to the implications for our work and some specific valve technologies.

\section{Supersonic Expansion}\label{secsuperexp}

Suppose that an ideal gas is trapped in a certain volume $V_1$ and at a certain pressure $P_1$, when one of the walls enclosing the volume is suddenly removed, and the gas is allowed to expand into a much larger volume $V_2$.
To determine the long term properties of the system, one can use the fact that neither heat nor work were transferred out of the gas, so that its energy $E$ must remain fixed, and therefore also its temperature, following the well-known intrinsic formula relating the temperature of an ideal gas to its internal energy, $E = C_VNT$.
It would follow that $P_2$ = $P_1\cdot V_1/V_2$, and that the entropy would be significantly increased.
To determine by how much, we can use the thermodynamic relation for the internal energy $E$ of the system:
\begin{equation}
dE = TdS - PdV,
\end{equation}
which since $\Delta E=0$ for this process since it transfers , leads us to:
\begin{equation}
\Delta S = \frac{1}{T}\int PdV = \frac{1}{T}\int\limits_{V_1}^{V_2}\frac{P_1V_1}{V}dV = Nk_B\log{\left(\frac{V_2}{V_1}\right)},
\end{equation}
Since $P(V) = P_1V_1/V$ for fixed $T$ and since $P_1V_1 = Nk_BT$.
This is also intuitively valid, since the entropy ought to correspond to the logarithm of accessible micro-states.
With nothing else changing about the momentum distribution before and after the event, only the system size contributes to an increase in accessible micro-states, by an amount given by the ratio of volumes before and after.

Although the foregoing example is ubiquitous in the problem sets assigned to physicists and mechanical engineers during thermodynamics classes, the corresponding short term behavior is rarely if ever treated.
At first glance, it may seem that the short term behavior would be intractable to the methods of statistical physics, owing to the suddenness of the event and its obvious lack of reversibility.
To the contrary, much progress can be made once some additional details and assumptions are specified.
Suppose that the initial volume is allowed to expand into a much larger volume, but only through a rather small opening.
Suppose further that the expansion process is in a flow equilibrium, so that gas is constantly supplied to the initial volume in order to maintain a constant pressure, and gas is constantly pumped out of the back of the much larger volume.
This supposition includes what is known as the continuum hypothesis, which stipulates that the gas may be treated as a fluid of smoothly varying density on all scales, and essentially requires a short mean-free path and a suitably high local density~\citep[Sec.~1.2]{Batchelor1967}.
These assumptions create a very similar environment to the one realized in many molecular beam apparatuses, known as a supersonic expansion.
Now consider a small sub-volume $V_1$ within the high pressure volume, close to the small opening.
Since the speed of egress is fast compared to diffusion timescales, we may assume that this volume is isolated from the rest of the gas, and consider its fate as the molecules within follow the flow lines of the fluid.
As we follow the flow lines, this volume expands, performing work on the surrounding environment, but not exchanging any heat with it.
Paradoxically, the process is then adiabatic, despite the significant and non-zero $\Delta S$ calculated above for the long term behavior.
The work done during expansion primarily serves to accelerate the volume just ahead along the flow, and the magnitude of the acceleration is such that information in the form of pressure fluctuations is not able to keep pace with the expansion~\cite{Miller1988,StuhlThesis2012}.
This is the sense in which it is termed a supersonic expansion.
%Of course an adiabatic process should be reversible, and indeed, were one to continuously flow gas through a small hole into a vacuum chamber, provided the flow remained in the continuum regime and the chamber were somehow aerodynamically shaped so as to prevent turbulence, there is no conceptual reason the vacuum pumps could not be reversed and some suction applied to the gas reservoir so as to suck the flow back through the nozzle.

As the gas expands, it continues to work on the environment until it has exchanged all of its internal energy, in principle reaching $T=0$ and satisfying Bernoulli's flow equation by converting its initial enthalpy entirely to kinetic energy:
\begin{equation}
H_i = C_PNk_BT_i = \frac{N}{2}mv_f^2.
\end{equation}
For a monoatomic ideal gas, $C_P = 5/2$, and we have:
\begin{equation}
v_f = \sqrt{\frac{5k_BT_i}{m}}.\label{eqfinalv}
\end{equation}
The final velocities obtained from this equation for noble gas carriers are shown in Tab.~\ref{tabfinal}.
\begin{table}[t!]
\centering
\caption[Final Supersonic Expansion Velocities]{Final supersonic expansion velocities for various gases according to Eq.~\ref{eqfinalv}, using $T=300^\circ$K. For H$_2$ I have used the ideal gas heat capacity since rotations are hardly populated at room temperature. For UF$_6$, inclusion in this chart is largely fanciful in any case, but rovibrations are unlikely to be efficiently cooled so I have again used the ideal gas heat capacity.\label{tabfinal}}
\begin{tabular}{r|l}
Gas & Speed (m/s)\\
\hline
H$_2$ & 2490\\
$^3$He & 2030 \\
$^4$He & 1760 \\
Ne & 790 \\
Ar & 560 \\
Kr & 380 \\
Xe & 310 \\
UF$_6$ & 190
\end{tabular}
\end{table}
In practice, these speeds are not always obtained by the species of interest within a noble gas buffer, often due to velocity slippage in cases of significant mass imbalance between the carrier and the species of interest.
Another source of discrepancy comes from excess heating, either due to the mechanism of operation of the valve~\citep[Sec.~3.1.3.1]{SawyerThesis2010}, or the mechanism of generating the species of interest, see Sec.~\ref{secdisch} below.

As the gas rarifies, the continuum flow hypothesis breaks down, and the mean free path between collisions of molecules becomes long enough that the assumption of an isolated sub-volume of the flow becomes invalid.
What this means is that below a certain pressure, the gas can no longer perform work on itself during expansion, and a limiting nonzero expansion temperature is reached.
In practice, the remaining internal energy of the gas associated with this nonzero final temperature is not a significant perturbation on the final velocity of the beam, but it does influence the phase space density available for later use.
The remaining internal energy of the gas manifests most measurably as an extra spread on the final velocity obtained during the expansion.
Also known as the translational temperature, its direct calculation is possible but complex~\cite{Montero2017}, but it is not overly difficult to measure, at least approximately, using a pulsed source and measuring at a near and distant point along the propagation of the pulsed expansion.
When this is done with a miniaturized solenoidal Even-Lavie valve~\cite{Even2014}, the values shown in Tab.~\ref{tabtrans} are obtained for a selection of rare gases.
\begin{table}[t!]
\centering
\caption[Final Supersonic Expansion Temperatures]{Final supersonic expansion temperatures for noble gas carriers in an Even-Lavie valve according to Ref.~\citep[Fig.~12]{Even2014}, using $T=300^\circ$K.\label{tabtrans}}
\begin{tabular}{r|l}
Gas & Temp ($^\circ$mK)\\
\hline
$^4$He & 33 \\
Ne & 180 \\
Ar & 930 \\
Kr & 2400 \\
\end{tabular}
\end{table}
Various factors can influence this, and often clustering effects can prevent utilization of the narrowest possible distributions for a seeded beam, see discussion in~\citep[Page~47]{SawyerThesis2010}.

\subsection{Rovibrational Temperatures}

An additional effect requiring careful accounting is the degree of thermalization of the particles in their frame of reference after the expansion.
One piece of this is the degree of thermalization of the rotational and vibrational degrees of freedom of molecules seeded in the beam with the translational degrees of freedom.
Generally it is found that these degrees of freedom do not thermalize as well as translation, with vibrations thermalizing much worse than rotations.
Much attention has been given to this subject, particularly given its very clear manifestation in an experimental observable, namely the relative population of the various rovibrational levels as probed by laser induced fluorescence or other state selective means.
It has even been found that the rotational levels fail to be well described by any meaningful temperature, even towards the beginning of the expansion~\cite{Hulsman2001}.
Studies specific to hydrides and even to the hydroxyl radical have been performed~\cite{Belikov2001}. 

\subsection{Translational Disequilibrium}\label{transdissec}

Compared to rovibrational levels where nearly complete information is available, it is rather difficult to measure the translational temperature of the beam in the direction transverse (orthogonal) to the beam-line. 
Evidence exists for a disequilibrium between transverse and longitudinal frame of reference temperatures~\cite{beijerinck1981,Toennies1977,Miller1988}.
This is especially important for consideration of the effective initial size of the beam as seen by devices downstream.
In the continuum flow region of a supersonic expansion, the fluid obeys a kind of hyperbolic flow pattern, which has the property that after a few nozzle diameters flow-lines actually appear to originate from a single point~\cite{Miller1988}.
As the molecules leave the continuum regime and the cooling process halts at nonzero temperature, this breaks down, so that individual molecules no longer appear to originate from a single point.

A useful way to understand these phenomena is to work in transverse phase space, where the velocity of molecules in one of the directions transverse to the beam is plotted against their position in that same direction.
If the cooling were to continue indefinitely and all molecules appeared to originate from a single point, the transverse phase space distribution function describing the ensemble at a fixed large time $t$ after passing through the orifice would look like an infinitely thin diagonal line with infinite phase space density. 
This is because particles having some transverse coordinate $x$ after traveling time $t$, if they originated exactly at $x=0$, must then have $v$ given exactly by $v = x/t$.
The nonzero final transverse temperature contributes extra spread in the velocity direction over this infinitely dense diagonal line, which is relevant for the use of hexapole focusers as discussed in Chapter~\ref{chapter:hex}.
An example of transverse phase space generated by propagating molecules perfectly from a source point and then adding nonzero transverse temperature at a distance reflective of the transition out of the continuum regime is shown in Fig.~\ref{transversephase}.
\figdave{transversephase}{Transverse Phase Space from a Supersonic Expansion}{
This scatter plot samples a distribution function designed to be reflective of that which may be generated in an Even-Lavie valve, see Sec.~\ref{evenvsec}. 
The quitting surface, where continuum flow ceases, is assumed to lie $3$~cm distant from the $500~\mu$m nozzle, and a transverse temperature of $200$~mK in both transverse and longitudinal directions is used.}{\linewidth}
This distance is referred to as the quitting surface in the literature~\cite{Miller1988}.
In Fig.~\ref{transversephase}, the molecules are back-propagated after the extra temperature included at the quitting surface, so that the distribution shown can be thought of as an effective source distribution in the same plane as the actual nozzle.
The distribution is quite tall vertically, which is a reflection of the fact that the beam expands on the surface of a sphere, so that transverse velocities even approaching the forwards velocity magnitude of $810$~m/s in this case should in principle be populated.
In practice such large transverse velocities are unusable, since even with an arbitrarily powerful focusing device molecules with very large transverse velocities also have forwards velocities reduced relative to the better collimated portion of the beam.
For this reason, I use an unphysical directed-ness parameter in generating distributions such as the one shown in Fig.~\ref{transversephase}, which has a FWHM polar angle of $\pi/12$, since the other molecules are not relevant for subsequent focusing and deceleration.
Actual directed-ness of supersonic expansions tend to follow an angular distribution with polar angle $\theta$ according to $\cos^3(\theta)$~\cite{Miller1988}, and some claim increased directed-ness through the use of conical expansion regions~\cite[Fig.~11]{Even2015}.

\subsection{Experimental Transverse Temperature}

It is possible to access some of this information experimentally by observing the angular distribution of a beam, especially downstream of some additional collimating aperture.
In the same way that light diffracts through a pinhole, a beam with nonzero transverse temperature, when imaged on a far field detector after passing an aperture, will appear as a disk with a blurry boundary, where the blurriness is proportional to this transverse temperature.
Another way to say this is that the transverse temperature contributes a point spread function to the transfer function that images an aperture onto a detector plane downstream, see Fig.~\ref{blurryboundary} for an illustration of this.
\figdave{blurryboundary}{Molecular Beam Imaging PSF}{
Transverse temperature lends a point spread function (PSF) to the image of apertures in the detector plane of a molecular beam. 
These distribution functions reflect a $20$~cm distance between the source and the aperture, and a $3$~mm aperture.
This investigation was first performed in June 2018, and relevant files may be found in a code appendix I intend to add at some point.}{14cm}
One way to measure this blurriness would be with an angle resolved, or adjustable position detector of some kind, but an easier alternative is to translate the distance between the source and aperture, thereby increasing and decreasing the size of the blurry spot in the detector plane.

We detected this effect during our work on skimmer cooling~\cite{Wu2018}, where the skimmer, a device designed to collimate a beam with minimum interference described in Chapter~\ref{chapter:skimming}, serves as the aperture. 
An important caveat is that any aperture has the potential to interfere with the beam, so that the blurriness observed could be an artifact caused by the aperture, and not a true representation of the original beam.
Data from~\citep[Fig.~2c]{Wu2018} are repeated here in Fig.~\ref{neontranslate} with an expanded discussion of these transverse effects that is not available in~\cite{Wu2018}.
\figdave{neontranslate}{Inferring Transverse Temperature from Spot Size}{
(a) Time of flight traces taken with varying distance between the valve and the aperture, with blue closest ($1.8$~cm) and red farthest ($4.8$~cm). 
The area under the curves are computed, leading to the data-points in the next panel. 
(b) Fit attempts allowing a variable fraction of the population in a second, warmer temperature distribution.
}{14cm}
For a suitably small transverse temperature, we would expect that once the valve is close enough to the aperture that the detector area is fully illuminated by the beam, signal no longer increases as a function of reduced valve skimmer coordinate.
This is at odds with the data shown in Fig.~\ref{neontranslate}, where the signal is seen to increase even below a $3$~cm valve-skimmer distance.
At this valve-skimmer distance, considering the $3$~mm skimmer opening and $40$~cm detector distance, the zero temperature spot size of the beam would be a $4$~cm diameter circle, well overlapping our fast ion gauge detector ($6$~mm x $18$~mm).
By adopting the assumption of a bimodal distribution, discussed in~\cite{beijerinck1981,Miller1988}, and varying the fraction assumed cold and the hot temperature, various fits are obtained as described in the legend of Fig.~\ref{neontranslate}b.
The cold fraction is assumed to be at $200$~mK, the measured longitudinal temperature, which well overlaps the detector at all valve-skimmer distances included here and essentially contributes only a baseline offset for the fitting.
The warm fraction is assumed responsible for the upturn at closest distances, leading to the various fractions and temperatures shown in the legend.
The temperatures required for obtaining any agreement with the experimentally observed upturn are not surprising.
In~\citep[Abstract, 3rd Paragraph]{beijerinck1981}, effective transverse source size for the hot distribution are claimed to be $4-8$ times that of the cold distribution, implying a temperature larger by that factor squared.
The fraction populating the hot distribution is much more surprising, with~\cite{beijerinck1981} reporting only $30-50\%$.

The primary alternative hypothesis is that the results obtained in Fig.~\ref{neontranslate} are not a pure reflection of the transverse properties of the supersonic expansion, but rather are the result of some degree of interference effects from the aperture itself.
The surprising thing here would be that the interference effects actually lead to an increase in signal, whereas the usual result of interference is only the reduction of signal.
It could be that this is a compound effect of some kind, for example that the close proximity of the valve may reduce interference with the aperture in a certain range. 
This could be caused by a favorable match between the angle of the conical skimmer faces and the position of the source, leading to a reduced cross section presented to the beam by the skimmer.
Another possibility is that ions generated in the discharge, see Sec.~\ref{secdisch}, have a cleaning effect that enhances the performance of the cryogenic skimmer.
This is not out of the question, as in-situ noble gas ion cleaning techniques are a legitimate surface preparation strategy~\cite{Hite2012}

\section{Pulsed Valves}

Pulsed valves solve what is otherwise a significant dilemma for a vacuum system, and enable significant improvements in brightness compared to their continuous analogs.
The key challenge for a pulsed valve is to rapidly transition from leak-tight to fully formed, low impedance flow.
Many strategies exist for achieving this, and many have been utilized right here in Jun's OH experiment.
A current loop, solenoid, and drum-mode piezo valve have been used, well described in~\cite[Sec.~3.1.3.1-3]{SawyerThesis2010}.
The latter proved particularly well suited for stable long-term operation, and continued to serve our needs well into 2017.
An impeccable  description of this valve is available here~\citep[Sec.~3.4.1.1]{BrilesThesis2015}, and of course its original publication~\cite{Proch1989}.
In what follows, I focus on two pulsed valves with which I have played a more direct role.

\subsection{Miniature Solenoid}\label{evenvsec}

Towards the middle of my thesis work, a decision was made to lengthen the decelerator and switch to a lighter Neon carrier gas.
The use of Neon opens up many opportunities for possible increases in source brightness, at the expense of increased beam forward velocity.
Neon has a lower viscosity, reduced propensity for clustering, colder final temperature, and reduced velocity slippage with OH due to the more comparable mass.
The reduced clustering issues make it possible to consider running at dramatically increased pressures, and also heating the valve to run at correspondingly enhanced water pressures.
The use of high pressures has an important effect on the operating mechanism for a pulsed valve, since this can significantly increase the mechanical force which must be exerted in order to open the valve.
This seems particularly problematic for piezo valves, which seem already somewhat limited in throw and force.
For this reason, together with the change to a longer decelerator, we decided to purchase an Even-Lavie valve~\cite{Even2014,Even2015}, considered to be a kind of gold standard in high density pulsed molecular beams, although with more challengers of late~\footnote{David Nesbitt claimed that this valve was comparable~\cite{Irimia2009} in email communication to Hao, Jun, and I}.
Refer to Fig.~\ref{evenlaviemounted} for a labeled view of the valve together with a custom designed and self-manufactured mounting structure.
\figdave{evenlaviemounted}{Mounted Even-Lavie Valve}{
The Even-Lavie valve, mounted in a thermally stabilized aluminum rig.
(i) The stainless steel body of the valve itself protrudes from the front of the aluminum rig.
(ii) This vertical aluminum cylinder is in fact a tube, containing within a water reservoir consisting of an NPT fitting with glass fiber filter papers inside for localizing some liquid water vapor.
(iii) A half-closed cylindrical holder for the valve body sits down here, and connects to the valve via hose-clamps. These hose clamps tighten onto copper sheets which help transfer heat to the upper aluminum slices which surround the valve for optimum thermal equilibration.
(iv) The valve mounts up to a VCR connector, by means of a VCR plug with 1/4"-28 threaded hole feature. This VCR connector formerly served as a gas line for the previous piezo valve.
(v) A frighteningly frustrating eight-pin mini-CF connector transfers wiring for two thermistors and two heaters. Teflon tubes prevent unintentional grounding between pins.
(vi) This Teflon block is original to the valve, and strain-relieves the feedthroughs for the wires which actuate the solenoid.
(vii) This clamp tightens the aluminum cylinder onto the water reservoir within (the cylinder has a cut to allow flex). A small vacuum compatible heater (Minco) is mounted underneath the clamp.
(viii) The leads to an Omega thermistor are connectorized to a teflon wire twisted pair and strain relieved via stainless steel wire to the hose clamp. The thermistor is glued into the upper aluminum cover piece on the front, to the left of this label and directly above the valve nozzle. An identical thermistor connects to the back of the water reservoir.
(ix) High voltage RF feedthrough for the dielectric barrier discharge.
(x) Ultra-thin 1/16" Swagelok gas line for minimum dead volume at high pressure.
}{10cm}
The valve utilizes a highly miniaturized solenoid type design, which actuates a ferromagnetic plunger that actually maintains its magnetic susceptibility down to cryogenic temperatures, allowing the valve to continue to function at those temperatures and opening up new possibilities for Helium beams and other physical chemistry pursuits.
It also boasts operation up to extreme pressures of $100$~bar, potentially a game changer for high brightness generation of OH radicals.

The purchase of the valve was not a pleasant experience, and involved over a year of waiting and much haggling about paperwork and international agreements.
The performance of the valve was also somewhat lackluster, with OH brightness in the source chamber actually only half that achieved with Neon in our previous piezo valve.
Tuning the temperature of the system did not yield any significant change in the performance of the system either, see Fig.~\ref{eventemp}.
\figdave{eventemp}{Even-Lavie Yield verse Temperature}{
Free flight of OH radicals past a detection region in the source chamber as a function of the temperature of the valve housing.
An optimum is found close to $40\,^\circ$K, only $\sim 20\%$ above the room temperature value.
}{\linewidth}
Increased temperature and increased partial pressure of water would have been necessary to capitalize on the possible gains of a higher pressure source region.
Great care was taken to optimize all relevant parameters, including discharge frequency, length, and delay; valve opening current and time; filament position and current.
Many plots similar to that in Fig.~\ref{eventemp} may be found in the data folders and logbooks from May-June of 2017.
On the bright side, the valve was successful in delivering an impressively low total gas load, and an impressive flux of Neon atoms, potentially a crucial enabler of our skimmer cooling results~\cite{Wu2018}.
In addition, the low translational temperature of Neon is essential for our as yet unrealized dream of studying collisions between Neon and OH in the midst of the beam~\citep[Chapter~7]{WuThesis2019}.

\subsection{Discharge}\label{secdisch}

For working with OH, or any other free radical, it is essential to integrate together with the valve body a mechanism for generating the radicals during the supersonic expansion process.
Generally, their reactive lifetime is too short to enable any other solution, and therefore a delicate balance arises between generating them far enough into the expansion process so that they do not have time to react, but not so far that they don't have time to be thermalized to the buffer gas during the expansion.
Various techniques exist, including both radio frequency and pulsed electric fields, and also UV photolysis of some precursor species.
All three of these have been attempted on this experiment, with the use of pulsed electric fields having been used over the majority of the experimental history prior to the Even-Lavie valve, which was purchased together with a so-called dielectric barrier discharge radio frequency source.
One exception was an attempt made at using photolysis of HNO$_3$, described in~\citep[Sec.~3.1.4.2]{SawyerThesis2010}, which produced only half the output of the piezo valve, and led to some rather hazardous laboratory conditions~\footnote{Private Communication with Brian Sawyer during DAMOP 2019.}\!.

A surprising observation we made during the testing of the Even-Lavie valve was that production efficiency of OH radicals continued to improve well beyond the recommended operating point of the device, yielding an additional factor of two by going beyond the maximum recommended RF discharge of $1.6$~kV or so.
This suggested to us that the RF discharging mechanism of the Even-Lavie valve could be the primary limiting factor associated with its performance for our operation.
This led us to be quite interested in the possibility of using a DC discharge with a high pressure Neon source, something which has at this point not yet been tested.

\subsection{Lorentz Force Opening Mechanism}

The Lorentz force exerted on a wire by a magnetic field containing the wire can also be exploited for operating valves.
A key advantage of this is the reduced obstruction of the required hardware for gas flow.
This style was designed in Nijmegen in the Netherlands~\cite{Yan2013}, and a similar design was undertaken in our shop, though not yet completed or tested at the time of this writing.
A primary motivation of the undertaking was the lead time of the Even-Lavie valve, but when this at last arrived, the effort was dropped.
Based on the performance of the Even-Lavie valve however, the possibility of operating this valve remains interesting, and a potentially viable way of actually leveraging the possibilities in principle afforded by having switched to a supersonic expansion in Neon.

The JILA valve design was performed by Kyle Thatcher, and contains many nice optimizations relative to the original~\cite{Yan2013}, especially an externally tunable positioning of the sealing device, which is shown in Fig.~\ref{thatchervalve}.
\figdave{thatchervalve}{Lorentz Wire and Sealing Device of a Nijmegen-Style Valve}{
This high spring coefficient aluminum alloy wire sits within a large magnetic field and pulls open a valve due to the Lorentz force resulting from a brief $1000$~A current pulse.
The design was taken from~\cite{Yan2013} and manufactured by Kyle Thatcher in the JILA machine shop.
}{\linewidth}
The valve also features careful material engineering to enable operation into the $100$~bar regime, although the relation between opening current and stagnation pressure will need to be explored.
A traditional DC discharge stack sits in front, and is also capable of higher voltages than usual, motivated by our experience with the Even-Lavie valve.

\section{Conclusion}

The major undertakings that have occurred during this and all theses of the OH experiment demonstrate the challenge and difficulty associated with maximizing the performance of a beam source.
A new effort associated with leveraging the seemingly promising characteristics of a Neon seeded beam has begun, with minimal results thus far.
One thing worth noting is that switching back to the older piezo valve should at least provide a decent two to four fold gain relative to the performance of the Even-Lavie valve obtained thus far, although this valve may play much less well with the cryogenic techniques discussed in the next chapters because of its increased flux.
Should an effort be deemed worthwhile, the new JILA version of the Nijmegen valve could serve as an excellent starting point for exploring the frontiers of optimum OH production in the Neon regime.


\ifx\justbeingincluded\undefined
\bibliographystyle{unsrtDR}	% or "siam", or "alpha", etc.
\bibliography{allrefs}		% Bib database in "allrefs.bib"
\end{document}
\fi
