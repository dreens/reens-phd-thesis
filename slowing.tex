\chapter{Optimized Deceleration}

Over the past two decades, Stark deceleration has enabled groundbreaking collisional~\cite{Sawyer2011,Kirste2012,Gao2018} and spectroscopic~\cite{Veldhoven2004,Hudson2006,Lev2006,Fast2018} studies of a variety of species~\cite{VanDeMeerakker2012}. 
Subsequent trap-loading greatly enhances interrogation time for such studies~\cite{Sawyer2008} and opens the door for further cooling and manipulation~\cite{Stuhl2012evap, Reens2017}. 
Zeeman deceleration has developed in parallel and enabled similar achievements for paramagnetic species. 


\section{Decelerator Geometry}

Much like related work in charged particle accelerators, a neutral particle decelerator is required to satisfy two key principles:
\begin{enumerate}
\item A particular candidate particle known as the synchronous molecule must be manipulated as desired by the device, e.g. slowed from $800--50$~m/s.
\item Other particles of similar initial conditions to the synchronous molecule must also possess similar final conditions.
\end{enumerate}
This second principle is known as phase stability, and is crucial to the overall performance of the device.

Neutral particle decelerators face an incredible setback relative to charged particle accelerators, in that it is vastly harder to apply forces on them.
Specifically, the device I will soon describe can accelerate hydroxyl radicals at $\sim 200$~km/s/s. 
If hydroxyl cations were instead placed in the largest electric fields generated by the device of about $100$~kV/cm, they would experience an acceleration given by:
\begin{equation}
\frac{q_e\cdot E_\text{max} }{m_\text{OH}} = 5.5\times 10^9 km/s/s.
\end{equation}
This constitutes a factor of over ten million. 
But is this truly fundamental, or technically limited? 

Applying forces on neutral particles requires the application of gradients in electric field magnitude, and the magnitude of these gradients depends on the miniaturization of the geometry the electrodes.
It becomes challenging to develop a truly fair comparison, but underneath all of the details about electrodes and geometries, electric fields are always generated by charged particles, regardless of whether they are conduction band electrodes in some material.
The force on a neutral particle by a charged one scales as $r^4$, while that between charged ones scales as $r^2$.

In any case, once the factor of ten million setback is taken into account, neutrals have the nice property that only the magnitude of the field influences them, and not the direction. 
This makes it possible to generate large magnitude electric fields with the direction of the field orthogonal to the beamline, which is precisely the trick utilized in the first Stark decelerators~\cite{Bethlem1999}. 

\begin{figure}


\end{figure}

Diagrams such as that shown in Fig.~\ref{decelcartoon} are useful in understanding the operation of the device.
Molecules approach a region of strong electric field, exchanging kinetic energy for internal potential energy in the case of those whose quantum state features a positive Stark shift, the only ones capable of being manipulated by such a device.
At some point, the strong electric fields are turned off, and a different region of strong electric field is turned on further along the beamline.
If the strong fields are turned off before the synchronous molecule reaches the strongest fields, i.e. partway up the hill, phase stability is obtained, because a molecule that is ahead of the synchronous one will therefore exchange a greater quantity of kinetic energy for potential energy by climbing farther up the hill before the turn-off event.
This molecule thus experiences a force restoring it to the location of the synchronous molecule.
Of course if a molecule is too far ahead, it may pass all the way through the region of highest field prior to the turn-off event. Such a molecule is no longer phase stable and will travel farther and farther from the synchronous molecule.
Thus there is a tradeoff between the deceleration capability of the device and the volume of its phase stable region.

\section{Alternative Charging Configurations}

Alongside the history of achievements enabled by Stark deceleration runs a parallel ongoing saga surrounding their efficient operation. 
Many important steps have been made, not only in understanding the flaws of the canonical pulsed decelerator~\cite{VanDeMeerakker2006,Sawyer2008a}, but also in addressing them through the use of overtones~\cite{VanDeMeerakker2005a,Scharfenberg2009}, undertones~\cite{Zhang2016}, or even mixed phase angles~\cite{Parazzoli2009,Hou2013}. 
Even with these advances, the outstanding inefficiencies of the pulsed decelerator, particularly with regard to transverse phase stability, have motivated alternative geometries such as interspersed quadrupole focusing~\cite{Sawyer2008a} and traveling wave deceleration~\cite{Osterwalder2010,VandenBerg2014,Fabrikant2014}. 
Although traveling wave deceleration takes a strong step in the right direction toward truly efficient operation, it comes with costs in system complexity and high voltage engineering. 
These costs can be partially addressed by the use of combination pulsed and traveling wave devices~\cite{Quintero-Perez2013}, or even using traveling wave geometry with pulsed electronics~\cite{Hou2016,Shyur2017}. 
Others continue to pursue brand new geometries aiming to enhance transverse acceptance without abandoning more reliable pulsed electronics~\cite{Wang2016}. 

\section{Trap Oscillations}

\section{HV Isolation Strategies}

\section{Manufacturing Considerations}