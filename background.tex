\ifx\justbeingincluded\undefined
\input chappreamble.tex
\input foolthesis.tex
\fi

\chapter{Background}
\label{chapter:Background}

All of our ability to successfully exploit and manipulate OH radicals is underpinned by a robust understanding of its internal structure undergirded by decades of careful spectroscopy.
Early spectroscopic results are still directly useful in determining where to set detection lasers~\cite{Meerts1975}, although many of these parameters are now more usefully catalogued in the NIST webbook~\cite{Huber2018}.
As in many areas of modern quantum physics, careful understanding begins with classical Hamiltonians followed by quantization and then a perturbative approach to better and better models of the system. 

\section{Molecular Hamiltonians}

Much like the well studied progression from classical descriptions of electrons orbiting a hydrogen nucleus to semiclassical angular momentum quantization and on to the exact Hydrogen quantum mechanical wave-function solution~\citep[Sec.~4.2]{Griffiths2018}, molecules may be understood beginning with the simplified quantum mechanics of rigid rotors and springs.
The former end up applying well to the rotational degrees of freedom of the molecules, and the latter to their vibrational degrees of freedom.


\subsection{The Rigid Rotor}

For the rigid rotor, we imagine some extended rigid body with moment of inertia $I_z$ about its center of mass in a particular direction $z$, which may be calculated in the usual way:
\begin{equation}
I_z = \int\limits_V\rho r^2 dAdz,\label{izsimp}
\end{equation}
where the integral covers the whole volume of the body with density $\rho$, and $r$ gives the distance from the infinitesimal volume element to the axis parallel to $z$ through the center of mass of the body.
Classically, lacking any environmental coupling, this body may indefinitely rotate about its axis of rotation with any angular velocity $\omega$, and posses a rotational kinetic energy given by:
\begin{equation}
U=\frac{1}{2}I\omega^2,
\end{equation}
and an angular momentum given by:
\begin{equation}
L= I\omega
\end{equation}
Semiclassically, we require the angular momentum to be quantized in units of $\hbar$, leading to the diophantine equation:
\begin{equation}
L = l\hbar = I\omega,
\end{equation}
with $l$ an integer, so that the allowed values of the angular velocity and energy become:
\begin{equation}
\omega = l\hbar/I, \qquad U = \frac{1}{2I}l^2\hbar^2
\end{equation}
with the natural consequence that for larger inertia $I$, the spectrum of allowed rotations becomes increasingly continuous.
This turns out to differ only slightly from the true situation, at least as it pertains to the specific case of linear rotors and diatomic molecules, where we find:
\begin{equation}
U = \frac{1}{2I}(l^2 + l)\hbar^2.
\end{equation}

It is instructive to follow this through more thoroughly~\citep[Sec.~2.8]{Brown2003}.
For the case of a linear rotor, where the body's mass is essentially confined to a single axis in space, a single momentum of inertia $I$ characterizes the system, and for the even simpler case where the mass is located at only two points separated by $R$, we can evaluate Eq.~\ref{izsimp} but with the following density distribution:
\begin{equation}
\rho(r)=m_1\,\delta\!\left(r-R\frac{m_2}{m_1+m_2}\right)+m_2\,\delta\!\left(r+R\frac{m_1}{m_1+m_2}\right),
\end{equation}
where $m_1$ and $m_2$ are the two masses separated by a distance $R$, and the values in the delta functions create the situation where the radial coordinate $r$ attains its zero value at the center of mass of the distribution.
Evaluating Eq.~\ref{izsimp} then leads to the result:
\begin{equation}
I = \frac{m_1m_2^2+m_2m_1^2}{(m_1+m_2)^2}R^2 = \frac{m_1m_2}{m_1+m_2}R^2 = \mu R^2,\label{eqdefI}
\end{equation}
where $\mu$ is the reduced mass of the system.
Now the Hamiltonian for this system is simply the rotational energy:
\begin{equation}
\hat{H} = \frac{1}{2\mu}\nabla^2
\end{equation}
which upon moving ahead with the Schr\"{o}dinger Equation, gives the spherical harmonic eigenfunctions:
\begin{equation}
\hat{H}Y_{lm}(\theta,\phi) = E_lY_{lm}(\theta,\phi),\qquad E_l = Bl(l+1),\qquad B=\frac{\hbar^2}{2I}.\label{eqdefb}
\end{equation}
Here $B$ is referred to as the rotational constant, and $\theta$, $\phi$ give physicist spherical coordinates relative to an axis orthogonal to the axis of the linear rotor, i.e. an axis along which the system has angular momentum $I$, \emph{not} the axis of the rotor itself.
It is interesting to note that the wave-functions associated with these eigen-energies are quite bizarre from the perspective of classical physics, especially given that relative to the quantum mechanics of the light electron, these coordinates describe the locations of heavy atomic nuclei.
For example, in the case of $l=0$, quantum mechanically the molecule is in a spherically symmetric state with equal probability of finding either nuclei anywhere on the surface of the two spherical shells correspending to their generally different radii away from the center of mass.

It is interesting to compare the values of $B$ calculated na\:{i}vely from Eqs.~\ref{eqdefb},~\ref{eqdefI} to the spectroscopically determined values, as shown in Tab.~\ref{bbbtable}. 
Very good agreement is obtained, at least when using the spectroscopically determined internuclear distance, which is therefore not quite fair as far as theory experiment predictions are concerned.
Nonetheless, it is reasonable to conclude that the rigid linear rotor model is a good description of the situation for diatomic molecules.
The bulk of the remaining deviations relate to centrifugal affects, by which I broadly refer to the interaction between the rotational and the vibrational components of the Hamiltonian, a nice segue into the following section.

\renewcommand{\arraystretch}{1.2}
\begin{table}[t!]
\centering
\caption[Molecular Rotational Constants]{
Parameters leading to the value of the rotational constant $B$ are shown, together with its semi-classically inferred value.
Spectroscopically determined values agree quite well. Radii and spectroscopically determined rotational constants from~\cite{Huber2018}, except RbCs from~\cite{Fellows1999} and YbCs from~\cite{Meyer2009}.
\label{bbbtable}}
\begin{tabular}{ C{2cm} | C{2cm} | C{2cm} | C{2.2cm} | C{2cm} |}
Molecule & $\mu$~(amu) & $R$~(\AA) & $B_\text{semi}$~(GHz) & $B$~(GHz) \\
\hline
H$_\text{2}$		& 0.5 & 0.74 & 1850 & 1820 \\
OH 		& 0.94 & 0.97 & 573 & 567 \\
LiH 		& 0.88 & 1.60 & 227 & 225 \\
YO   		& 13.6 & 1.79 & 11.6 & 11.6 \\
CaF 		& 12.9 & 1.97 & 10.1 & 10.1 \\
SrF		& 15.6 & 2.08 & 7.52 & 7.51 \\
YbF 		& 17.1 & 2.02 & 7.24 & 7.22 \\
RbCs 	& 51.9 & 4.4 & 0.503 & 0.497 \\
YbCs	& 75.2 & 5.4 & 0.230 & 0.207 \\
\end{tabular}
\end{table}

\subsection{The Vibrational Hamiltonian}

In much the same way as the canonical simple harmonic oscillator problem in quantum mechanics, at its core the vibrational Hamiltonian may be represented as a harmonic potential describing the energetics of the internuclear distance of the molecule.
This potential may be specified in a more formal way beginning with the full Hamiltonian describing all of the electrons and nuclei in a molecule, and then separating the electronic and nuclear portions based on a hierarchy of timescales known as the Born-Oppenheimer approximation~\citep[Sec.~8.1]{Atkins2005}.
Essentially, the nuclear positions are taken as parameters to the computation of the electronic wavefunctions, and so by parametrically solving the electronic Hamiltonian for all nuclear distances and rotations, the energetic landscape governing the dynamics of the nuclei emerges.
It is therefore somewhat difficult to directly specify the form of the interatomic potential experienced by the nuclei, but the results from this procedure turn out to be closely represented by the Morse potential:
\begin{equation}
V(x) = \left(1-e^{-\alpha x}\right)^2.
\end{equation}
This potential has a minimum at $x=0$, tends to infinity as $x\rightarrow-\infty$, and to $1$ as $x\rightarrow\infty$. For small $x$, expanding near zero gives a harmonic form as we would expect:
\begin{equation}
V(x) \approx \left(1-(1-\alpha x + \alpha^2 x^2 / 2 - ...)\right)^2 \approx \alpha^2 x^2.
\end{equation}
This potential is therefore an anharmonic oscillator, with the nature of the anharmonicity being such that the restoring force weakens for larger deviations, so that higher energy levels are more closely spaced.
Another important property of the Morse oscillator is that only a finite number of levels are supported, which is also the case for true molecular potentials, unlike electronic energy levels.


\section{Angular Momentum Coupling}

In addition to the nice descriptions available in earlier theses, I'd like to leave my mark by including for the first time animations of molecular angular motion that are faithful to the real thing except for time rescaling.


\section{Molecular Collisions}



\ifx\justbeingincluded\undefined
\bibliographystyle{unsrtDR}	% or "siam", or "alpha", etc.
\bibliography{allrefs}		% Bib database in "allrefs.bib"
\end{document}
\fi
