\ifx\justbeingincluded\undefined
\input chappreamble.tex
\fi

\chapter{Hexapole Focusing}
\label{chapter:hex}

And it came to pass that all those molecules who contrived to survive the great segregation described in the previous chapter, were set about a grueling and strenuous exertion of the mental faculties. Indeed, the focus requested of them was of such a severity as to literally bend their wills and their trajectories. And in more modern terminology, the subject of this chapter is the focusing of molecular beams for optimized coupling between their source and any further scientific requirements downstream.

\section{Phase Space Matching}

To understand the action and possible usefulness of a hexapole, it is essential to become familiar with thinking and working in phase space.
By phase space I simply mean the two dimensional space spanned by the position of molecules in the ensemble on one axis and their velocity on the other.
Technically speaking phase space ought to refer to the position-momentum space, but for SI units and for our very light particles velocity makes much more sense.
The usefulness of working in phase space stems from the straightforward correlation between the various transformations that may be applied to the ensemble and how they are manifested in phase space, as is summed up in Tab.~\ref{phasespacetable}.
Free propagation results in molecules with larger velocity coordinate

\newcommand{\littlefig}[1]{\vspace{-3mm}\newline\includegraphics[clip,trim=1.5cm 2cm 1.2cm 1.7cm,width=3cm]{#1}\vspace{-1mm}}
\renewcommand{\arraystretch}{1}
\begin{table}[t!]
\centering
\caption[Phase Space Manipulations]{
Phase space manipulations are described and visualized. Velocity is the vertical axis, Position the horizontal. Gray shows the ensemble before the action, black afterwards. The first four actions are applied subsequently, the last two are applied directly after initialization.\label{trappingtable}
}
\label{tab:rates}
\begin{tabular}{ C{5cm} | C{6cm} | C{5cm} }
Action & Effect & Visualization \\
\hline\hline
Initialization 			& Some Approximately Gaussian Distribution 	& \littlefig{PSM1.png} 	 \\
\hline
Free Propagation 		& Upper region to the right, vice versa		& \littlefig{PSM2.png} 	 \\
\hline
Hexapole Focusing 		& Clockwise Rotation					& \littlefig{PSM3.png} 	 \\
\hline
Additional Free Propagation	 	& Spatial Density Revival 					& \littlefig{PSM4.png} 	 \\
\hline
Free + Hex + Long Free   	& Velocity Compression 					& \littlefig{PSM5.png} 	 \\
\hline
Free + Hex + Short Free 	& Spatial Compression 					& \littlefig{PSM6.png} 	 \\
\hline
\end{tabular}
\end{table}


\section{History}

\section{Modernity}


















\ifx\justbeingincluded\undefined
\bibliographystyle{unsrtDR}	% or "siam", or "alpha", etc.
\bibliography{allrefs}		% Bib database in "allrefs.bib"
\end{document}
\fi