\ifx\justbeingincluded\undefined
\input chappreamble.tex
\fi

\chapter{Hexapole Focusing}
\label{chapter:hex}

And it came to pass that all those molecules who contrived to survive the great segregation described in the previous chapter, were set about a grueling and strenuous exertion of the mental faculties. Indeed, the focus requested of them was of such a severity as to literally bend their wills and their trajectories. And in more modern terminology, the subject of this chapter is the focusing of molecular beams for optimized coupling between their source and any ensuing scientific apparatus.

\section{Phase Space Matching}

To understand the action and possible usefulness of a hexapole, it is essential to become familiar with thinking and working in phase space.
By phase space I simply mean the two dimensional space spanned by the position of molecules in the ensemble on one axis and their velocity on the other.
Technically speaking phase space ought to refer to the position-momentum space, but for SI units and for our very light particles velocity makes much more sense.
The usefulness of working in phase space stems from the straightforward correlation between the various transformations that may be applied to the ensemble and how they are manifested in phase space, as is summed up in Tab.~\ref{phasespacetable}.
Free propagation results in molecules with larger velocity coordinate deviating towards larger spatial coordinate, and vice versa.
I always work in the frame of reference of the ensemble, so that the centroid of the initial ensemble is assigned the $(0,0)$ coordinate.
The action of a hexapole is to create a spring-like harmonic restoring force in the transverse directions, in which case it may be proven that motion in phase space is circular, provided the velocity axis is scaled by the angular frequency of the harmonic restoring force into spatial units: $v` = v/\omega$.
Beginning with the spring-like restoring force:
\begin{equation}
\ddot{x} = -kx/m_\text{OH} = -\omega^2x,
\end{equation}
and obtaining the solution for the position and velocity under simple harmonic motion:
\begin{equation}
x(t) = \sin(\omega t + \phi),\qquad v(t) =\dot{x}(t) = \omega\cos(\omega t + \phi).
\end{equation}
If we now scale the velocity units by $\omega$, we obtain:
\begin{equation}
\biggl(x(t)\biggr)^2 + \left(\frac{v(t)}{\omega}\right)^2\quad = \quad\sin^2(\omega t + \phi) + \cos^2(\omega t + \phi)\quad =\quad 1.
\end{equation}
By appropriately choosing the frequency of the hexapole and the length of time spent within, it is possible to prepare the ensemble for a revival of the initial distribution after some additional free propagation afterwards, see the third and fourth rows of Tab.~\ref{phasespacetable}.

\newcommand{\littlefig}[1]{\vspace{-3mm}\newline\includegraphics[clip,trim=1.5cm 2cm 1.2cm 1.7cm,width=3cm]{#1}\vspace{-1mm}}
\renewcommand{\arraystretch}{1}
\begin{table}[t!]
\centering
\caption[Phase Space Manipulations]{
Phase space manipulations are described and visualized. 
Velocity is the vertical axis, Position the horizontal. 
Gray shows the ensemble before the action, black afterwards. 
The first four actions are applied subsequently, the last two are applied directly after initialization.
\label{phasespacetable}}
\begin{tabular}{ C{5cm} | C{5.5cm} | C{4.5cm} }
Action & Effect & Visualization \\
\hline\hline
Initialization 			& Some approximately Gaussian Distribution 	& \littlefig{PSM1.png} 	 \\
\hline
Free Propagation 		& Upper Region to Right, Lower to Left		& \littlefig{PSM2.png} 	 \\
\hline
Hexapole Focusing 		& Clockwise Rotation					& \littlefig{PSM3.png} 	 \\
\hline
Additional Free Propagation	 	& Spatial Density Revival 					& \littlefig{PSM4.png} 	 \\
\hline
Free + Hex + Long Free   	& Velocity Compression 					& \littlefig{PSM5.png} 	 \\
\hline
Free + Hex + Short Free 	& Spatial Compression 					& \littlefig{PSM6.png} 	 \\
\hline
\end{tabular}
\end{table}

Furthermore, it is possible to engineer the applications of hexapole rotation and free flights to achieve compression along one axis, accompanied by density preserving dilation in the opposite axis.
These results parallel the magnification or demagnification of an image achieved by a lens between an image and object plane depending on whether the lens is positioned closer to the image or the object plane.
One key distinction however between the behavior of an optical lens and that of a hexapolar lens is that the latter is almost never operated in the short-lens regime, which has a significant impact on how one thinks about its behavior.
Molecules are often rapidly transiting a hexapole in the longitudinal direction, which means that the device needs to be long in order to apply the desired rotation of the transverse phase space.
By carefully following an ABCD matrix approach, instead of a simple lens-maker equation for the locations of image and object planes, we obtain~\citep[Eq.~9]{Kirste2013}:
\begin{equation}
L_3 = \frac{L_1+\frac{1}{\kappa}\tan(\kappa L_2)}{L_1\kappa\tan(\kappa L_2)-1},
\end{equation}
where $\kappa=\omega/v$ is an inverse length unit directly related to the hexapole strength and inversely to the beam speed, and the $L_i$ are as described in Fig.~\ref{hexdefinitions.png}.
In the limit of small $L_2$, this equation can be reduced to something more familiar.
First, using $\tan(x)\sim x$,
\begin{equation}
L_3 = \frac{L_1+L_2}{L_1L_2\kappa^2-1}= \frac{L_1}{L_1L_2\kappa^2-1},
\end{equation}
and now inverting,
\begin{equation}
\frac{1}{L_3} = L_2\kappa^2 - \frac{1}{L_1},
\end{equation}
so that for $f_H=1/L_2\kappa^2$ we recover the lensmaker formula:
\begin{equation}
\frac{1}{L_1}+\frac{1}{L_3}=\frac{1}{f_H}.
\end{equation}
This focal length has the dependences we would expect, since it reduces (i.e. the lens strengthens) with larger $L_2$ or with larger $\omega$.

\figdave{hexdefinitions.png}{Hexapole Focusing Variables}{Sample trajectories of molecules focused by a hexapole of length $L_2$ after traveling $L_1$ to arrive and an additional $L_3$ afterwards.}{12cm}

This compression of velocity and dilation of position or vice versa is what is known as phase space matching, and is in principle useful for optimizing the transfer of molecules in an ensemble between a source and a device, or between devices.
To flesh this out a bit further, all the devices worked with in the laboratory can be characterized by a phase space acceptance, in principal a six dimensional region but in practice only ever considered two dimensions at a time.
The phase space acceptance typical of the Stark decelerator for example is indicated in Fig.~\ref{phasespaceexamp.png}, and discussed further in Chap.~\ref{chapter:slowing}.
Unfortunately the phase space generated by a supersonic expansion is rather difficult to ascertain, as discussed previously in Chapt.~\ref{chapter:sourcing}.
One thing which can be said for certain however is that the initial transverse velocity width generated by the supersonic expansion is much larger relative to anything else in the experiment.
This follows from simple intuition as the supersonic expansion only generates a mildly directed beam, which therefore expands significantly transversely away from the valve.
Projected onto transverse phase space, such an ensemble should have a variation of transverse velocity that is on the same order as the forward velocity of the beam, i.e. hundreds of m/s, compared with tens of m/s for the typical depths of devices made with laboratory electric fields.
Nonetheless, the valve may still generate a distribution that is quite narrow in its initial transverse spatial spread.
If the initial distribution is narrower spatially than the $2$~mm of the decelerator, there will be opportunities for an improved loading of the decelerator with appropriate magnification of the device designed to compress velocity and dilate position as in row 5 of Tab.~\ref{phasespacetable}.

\figdave{phasespaceexamp.png}{Transverse Phase Space Acceptance Examples}{Transverse phase space acceptance of a Stark decelerator operated in various modes. Each mode has a different characteristic ratio of velocity width to spatial width. Some colors are hard to perceive, but all have a width close to $2$~mm and varying heights between $10$ and $30$~m/s.}{10cm}

\section{History}

In the first generation of the experiment, a $5$~cm hexapole was employed and successfully enhanced the molecule number just afterwards by as much as a factor of $16$~\citep[Fig.~6]{Bochinski2004}.
What is less clear however is whether this enhancement really exceeds what could have been had by simply mounting the decelerator closer to the valve and skipping the hexapole entirely.
By the time Hao and I were running the second generation of the experiment~\cite{Sawyer2007}, the hexapole was only providing a $17\%$ effect, as shown in Fig.~\ref{nohexeffect.png}.
I can find no earlier record of the performance of this hexapole, so it is difficult to say whether this observation of no effect with the voltage on/off stems from some HV issue that developed later, from transition between Jordan and PZT valves, or an alignment issue or some other effect.

\figdave{nohexeffect.png}{No Hexapole Effect in Generation Two}{Three traces showing OH population after bunching with the hexapole set at different voltages. A scant $17\%$ effect is observed.}{14cm}

If it is indeed the case that the hexapole in the second generation decelerator never had a significant effect, there are a few possible explanations.
One relates to the influence of skimmer clogging, which can alter the trajectories of molecules so that they appear to have originated from the skimmer tip.
The skimmer tip is too close to the hexapole relative to its focal length for it to have much influence, hence the observation of little effect.
Another possibility is that the initial distribution of molecules in transverse phase space is simply much larger than the phase space acceptance of the decelerator, so that even with no hexapole and pure free propagation, the initial distribution still thoroughly overlaps the acceptance of the decelerator.
This explanation would even jive with the observed influence of the hexapole in~\citep[Fig.~6]{Bochinski2004}, since that observation was made with a detector directly after the hexapole.
This detector could have observed a legitimate hexapole induced spatial density enhancement, but entirely composed of molecules with transverse velocities above the threshold for decelerator acceptance.


\section{Modernity}

Against this dubious backdrop, we decided in the summer of 2018 to begin work on a new hexapole for use with the third generation decelerator.
Our motivations were quite different however from those that motivated the poorly performing hexapole of the second generation.
Instead of seeking to optimize coupling between the valve and decelerator, we were seeking to intentionally distance the two for the sake of reduced clogging potential, without sacrificing the gains in phase space density we had obtained by cryogenically cooling the skimmer.
This was largely motivated by observation of significant decelerator clogging, as described previously in Chap.~\ref{chapter:segregation}.
The new hexapole would this require cooling to cryogenic temperatures, so as to avoid the clogging observed in the decelerator.
In a sense, one can think of the cryogenic hexapole as comparable to cryogenically cooling the decelerator, only much more practically attainable.

I was able to rather quickly cycle through hexapole designs thanks to rapid manufacturing services and a well-integrated design for manufacturing approach.
I begin by describing features common to the two and a half hexapoles tried out over the past year.
Firstly, the obvious- a hexapole consists of six alternately charged poles, which generates close to the axis an electric field with the functional form:
\begin{equation}
\left|\vec{E}(r)\right| = 3V_0\left(\frac{r^2}{r_0^3}\right),
\end{equation}
where $V_0$ is the magnitude of the voltage to which each rod is charged, with alternating sign, and $r_0$ is the smallest radius which reaches from the axis to the surface of a pin.
For a molecule with a linear response to the electric field, this quadratic dependence of the electric field magnitude results in a harmonic potential, with the exception of the relatively small region close to the axis where the field is too small to polarize the molecules.
This exception causes a perturbation to the ideal focusing behavior, but is not very significant for OH~\citep[Fig.~2b]{Bochinski2004}, although more so for other species~\citep[Fig.~1]{Kirste2013}.
Specifically, with $V_0=13$~kV and $r_0=5$~mm, which lead to the typical field strengths characteristic of our traps and decelerators, electric fields up to $100$~kV/cm are generated, compared to $3$~kV/cm for polarizing the radical.

\subsection{Cryogenic Hexapole 1}

My first attempt at a cryogenic hexapole design leveraged some rather complex technologies.
Fearing the possibility of gas mediated discharge of the hexapole electrodes due to the high densities of molecules shortly after the skimmer, I planned to have the electrodes encased in sapphire tubes, so as to avoid the availability of conduction electrons at their surfaces.
These sapphire tubes would in turn sit within a copper block, for maximal thermal conductivity, mechanical alignment, and to provide additional adsorbing surfaces for the Neon.
The fields generated in the planned configuration at its intended voltages are shown in Fig.~\ref{hex1fields.png}.
In panel (a), note how the fields inside the sapphire are particularly small, a byproduct of its large dielectric constant close to $11$.
In panel (b), the electric fields along two axes are shown, one through the electrodes and the other between them.
A quadratic guiding curve is also shown, which corresponds to an angular trapping frequency for OH radicals of $\omega=20.6$~kHz.

\figdave{hex1fields.png}{Electric Fields in Hexapole 1}{(a) Electric field magnitudes and geometric size of the first generation hexapole, composed of 1/16" steel electrodes mounted within 1/8" sapphire tubes mounted within a 1" copper block. Note how the fields inside the sapphire are particularly small, a byproduct of its large dielectric constant close to $11$. (b) Field Strength (also kV/cm) verse distance off axis along various axes as described in the main text.}{\linewidth}

This all of course requires patterning of a honeycomb-like structure on the inside of a long copper block, something which may be readily achieved using a technique known as wire electric discharge machining, wherein a thin sacrificial wire at high voltage is used like a bandsaw blade, and is slowly moved through another metal.
Electric discharges between the wire and the metal being cut cause material to be removed, and the wire is constantly fed so as to avoid its disintegration.
This was achieved according to our specifications without any hitches by the CIRES shop, a shop located at the time of this writing in the chemistry department at CU.

Some of the thermal considerations pertinent to the design are illustrated in Fig.~\ref{hex1heatloads.png}.
The final mechanical design differs slightly from what is shown in this figure, but the thermal considerations remain the same.
The large copper block within which the hexapole sits has a skimmer affixed to the front and an aperture to the back, and a large clamp-like structure conveys sufficient thermal conductivity without requiring Indium solder.
Confirming this required careful consideration of the force-dependent thermal contact between the clamp and the hexapole, and also between the braid and the clamp.
Thermal conductivity across gold-gold and copper-copper pressure interfaces have been measured~\citep[Fig.~2.7]{Ekin2006}, and exceed grease joints above a certain pressure, see the discussion in~\citep[Sec.~2.6.4]{Ekin2006}.
As can be observed below in Fig.~\ref{hex1view.png}, we settle for a gold-copper joint, which naively would fall somewhere between the gold-gold and copper-copper performances.
Of course nothing is certain, see for example in~\citep[Fig.~5.8]{Hartwig2013} how mixing copper particles into epoxy can actually worsen its thermal conductivity in some temperature ranges.
It is also important to figure out how tight the clamp should be made, so as to benefit from the pressure-dependent thermal conductivity, without irreversibly deforming the copper parts.
I studied this using finite element analysis via Solidworks Premium Simulation package, see Fig.~\ref{hex1force.png}.
One interesting point worth mentioning is that the geometry of the device leads to a factor of $2\pi$ hoop stress amplification, where the force between the clamp and the cylinder is actually $2\pi$ times bigger than the force applied to the bolts.
This can be derived by conceptually dividing the geometry in half through the bolts and analyzing the static free body diagrams of each component, a standard exercise in statics.
The conclusion is that $1000$~kg on each of two bolts would give $240$~W/\,$^\circ$K for gold-gold and $3$~W/\,$^\circ$K for copper-copper, found by taking the $10\,^\circ$K $50$~kg data in~\citep[Fig.~5.8]{Hartwig2013} and multiplying up to $12000$~kg including the hoop stress amplification.
Of course, this brings to mind the question of validity of the linear scaling of conductivity all the way up to such large forces.
In~\cite{Berman1956}, extremely high loads were investigated, and although their cryogenic hydraulic ram apparatus failed catastrophically during the investigation, they were able to confirm that conductivity increased $5$0-fold for forces increasing $20$-fold up to $4700$~lb, a superlinear scaling to very high forces.

\figdave{hex1force}{Clamp Stress and Strain in Hexapole 1}{
(a) Stress resulting from clamp tightening. 
Annealed copper yields at only $70$~MPa or so, this indicates that $1000$~kg is on the upper end of what should be used on the bolts, and would accompany some yield on the edges of the block.
(b) Strain in the vertical direction only, so as to view what happens to the clamped hexapole mounting copper rod. 
Only $3~\mu$m radial compression is observed, small compared to the $3$~mm diameter sapphire tubes inserted into the rod.
}{\linewidth}

The emissivities guessed for the surfaces shown in panel (b) of Fig.~\ref{hex1heatloads.png} turned out to be optimistic, perhaps because they were approximate room temperature emissivities and not cryogenic, or maybe due to adsorbed gas layers further increasing emissivities.
A key challenge of the design pertains to how voltage is to be fed into the electrodes located inside the sapphire tubes.
A few iterations were made on the design before this was achieved with acceptable surface path-length reduction and suitable mechanical rigidity.
The HV distribution rings visible in Fig.~\ref{hex1heatloads.png} ended up having too much flexibility on their own, so plastic Ultem cuffs were glued directly onto the Sapphire rods, simultaneously increasing surface path length for arcs and adding the required rigidity.

\figdave{hex1heatloads.png}{Heat Conduction and Radiative Loads for Cryo-Hex 1}{(a) Conduction of thermal joints in W/$^\circ$K. In the final design, two braids were used, for a total conductivity closer likely above 1 W/K. (b) Radiative heat loads on exposed surfaces in mW.}{\linewidth}

An additional consideration for the hexapole is the matter of its alignment with the beamline.
There are two primary strategies by which satisfactory alignment may be obtained:
\begin{enumerate}
\item Contrive a way for alignment to be tweaked from outside of the vacuum chamber, so that final alignment may be guided by actual system figures of merit such as molecule yield after the hexapole. \label{hexitem1}
\item Provide in-vacuum fine-adjust knobs only, and engineer a protocol by which alignment may be obtained before pump-down, and not subsequently lost during either pump-down or cool-down. \label{hexitem2}
\end{enumerate}
Strategy~(\ref{hexitem1}) was employed for our cryogenic skimming work as described previously in Chap.~\ref{chapter:segregation}, but was deemed insufficient for this hexapole work, primarily due to unavailability of the requisite degrees of freedom.
Unlike the skimmer, which only really requires two translational degrees of freedom in order to achieve beam alignment, the hexapole has four alignment parameters, and potentially five depending on whether it is desirable to translate the device along the beam.
There seemed no straightforward way to achieve this across the vacuum envelope, especially given how far above the hexapole the tuning bellows sits.
Thus strategy~(\ref{hexitem2}) was pursued, and the problem of cool-down induced misalignment addressed via a flexible heat linkage\footnote{\href{https://www.techapps.com/}{Technology Applications Inc,} braid P6-502. Tyler Link, at the time of this writing, is always ready to send long emails about his products, at any hour of the day. Contact him directly for a product manual.}\!\!.

The principal challenge of utilizing a flexible heat linkage is finding a different way to achieve mechanical rigidity without violating thermal isolation.
For this purpose, I selected Ultem (or PEI, polyetherimide), a dimensionally stable engineering thermoplastic with poor thermal conduction and good mechanical properties.
Ultem rods 1/8" wide and 7/8" long can be seen in Fig.~\ref{hex1view.png}(i), which also illustrates other pertinent details of the mounting assembly.
Many polymers feature admirably low thermal conductivity, and the subject of polymer thermal conductivity is treated more fully in~\cite{Hartwig2013}, Chap.~5.
The same selection was once made for cryostats at the RHIC~\cite{Sondericker1994}.
In order to make estimations of the thermal conductivity for long, thin cylinders made of unfilled Ultem 1000 connected between room temperature and the cold apparatus, it is necessary to properly treat the thermal gradient which will necessarily build up within the material.
I address this first in a very general sense.
Suppose a material has a thermal conductivity $c(T)$ which varies with temperature $T$ and is connected as a cylindrical rod between reservoirs at temperatures $T_L$ and $T_H$.
The total heat transfer rate, $\dot{Q}$, must be the same through every slice of the rod, neglecting radiative heat loads, and will induce a temperature change through a thin slice $dx$ of the rod of the amount
\begin{equation}
dT = \frac{\dot{Q}}{c(T)\cdot A / dx},
\end{equation}
where $A$ is the cross sectional area of the rod. Rearranging and integrating:
\begin{equation}
\int\limits_{T_L}^{T_H}c(T)dT=\int\limits_0^L\frac{\dot{Q}dx}{A},
\end{equation}
\begin{equation}
\bar{c}\,\bigg|_{T_L}^{T_H} = \dot{Q}L/A,\label{qdoteq}
\end{equation}
where $\bar{c}$ is the antiderivative of $c$.
We may proceed using the data from~\cite{Sondericker1994} for slightly glass-filled Ultem 2100, which shows $0.04$~W/m$^\circ$K at $10\!~^\circ$K, $0.4$~W/m$^\circ$K at $300\!~^\circ$K, and an approximately straight line on a log-log plot in between.
It follows that $c(T)$ may be approximated as:
\begin{equation}
c(T) = \alpha T^\beta,\quad \beta = \frac{\log\left(\frac{0.4}{0.04}\right)}{\log\left(\frac{300}{10}\right)}=0.68,\quad \alpha = \frac{0.4}{300^{0.68}}=8.5\times10^{-3}\text{ W/m/}(^\circ\text{K})^{\beta+1},
\end{equation}
so that,
\begin{equation}
\bar{c}\,\bigg|_{T_L}^{T_H} = \frac{\alpha}{\beta+1} T^{\beta+1}\,\bigg|_{10^\circ\text{K}}^{300^\circ\text{K}} = 73\text{ W/m}\label{eq73},
\end{equation}
and finally with $A=7.9$~mm$^2$ and $L=20$~mm, we have rearranging Eq.~\ref{qdoteq}:
\begin{equation}
\dot{Q} = 73\text{ W/m}\times7.9\times10^-6\text{ m}^2/ (0.02\text{ m})=0.029\text{ W}.
\end{equation}
A table of thermal conductivity integrals with $T_L=4\,^\circ$K and various $T_H$ is provided in~\citep[App.~A2.1]{Ekin2006}. No value is included for Ultem, but the listed polymers have very similar values to that found above in Eq.~\ref{eq73}, for example Teflon is listed as $70$~W/m.
It is also worth pointing out that the conductivity integral between room temperature and $4\,^\circ$K only differs from that between room temperature and $50\,^\circ$K by $10\%$. 
This is symptomatic of the asymmetric distribution of thermal gradient, with most of it occurring in a very small section, and the majority of the plastic closer to room temperature since this is where the thermal conductivity is larger.

\figdave{hex1view.png}{Assembled view of Hexapole 1}{
A view of Hexapole 1 after assembly. 
(i) 1/8" x 7/8" Ultem rods to insulate but support the structure. 
(ii) Flexible heat linkage, $1.4$~W. 
(iii) Springs enabling fine positioning, four in total. 
(iv) Nichrome heater coil. 
(v) Blurry HV distribution ring barely visible. 
(vi) Invar plate spacers behind steel 1/4-28 nuts. 
(vii) Custom Skimmer, $3$~mm opening. 
(viii) Teflon kinematic cylinder visible for translating but allowing twist. 
(ix) Fine adjust knob, 3/16-100, mounted in a phosphor bronze bushing, torr-sealed into a steel block.
}{\linewidth}



\subsection{Cryogenic Hexapole 2}

\subsection{Cryogenic Hexapole 2.5}

\subsection{Cryogenic Hexapole 3}
















\ifx\justbeingincluded\undefined
\bibliographystyle{unsrtDR}	% or "siam", or "alpha", etc.
\bibliography{allrefs}		% Bib database in "allrefs.bib"
\end{document}
\fi