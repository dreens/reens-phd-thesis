\ifx\justbeingincluded\undefined
\input chappreamble.tex
\input foolthesis.tex
\fi

\chapter{Hexapole Focusing}
\label{chapter:hex}

And it came to pass that all those molecules who contrived to survive the great segregation described in the previous chapter, were set about a grueling and strenuous exertion of the mental faculties. Indeed, the focus requested of them was of such a severity as to literally bend their wills and their trajectories. And in more modern terminology, the subject of this chapter is the focusing of molecular beams for optimized coupling between their source and any ensuing scientific apparatus.

\section{Phase Space Matching}

To understand the action and possible usefulness of a hexapole, it is essential to become familiar with thinking and working in phase space.
By phase space I simply mean the two dimensional space spanned by the position of molecules in the ensemble on one axis and their velocity on the other.
Technically speaking phase space ought to refer to the position-momentum space, but for SI units and for our very light particles velocity makes much more sense.
The usefulness of working in phase space stems from the straightforward correlation between the various transformations that may be applied to the ensemble and how they are manifested in phase space, as is summed up in Tab.~\ref{phasespacetable}.
Free propagation results in molecules with larger velocity coordinate deviating towards larger spatial coordinate, and vice versa.
I always work in the frame of reference of the ensemble, so that the centroid of the initial ensemble is assigned the $(0,0)$ coordinate.
The action of a hexapole is to create a spring-like harmonic restoring force in the transverse directions, in which case it may be proven that motion in phase space is circular, provided the velocity axis is scaled by the angular frequency of the harmonic restoring force into spatial units: $v` = v/\omega$.
Beginning with the spring-like restoring force:
\begin{equation}
\ddot{x} = -kx/m_\text{OH} = -\omega^2x,
\end{equation}
and obtaining the solution for the position and velocity under simple harmonic motion:
\begin{equation}
x(t) = \sin(\omega t + \phi),\qquad v(t) =\dot{x}(t) = \omega\cos(\omega t + \phi).
\end{equation}
If we now scale the velocity units by $\omega$, we obtain:
\begin{equation}
\biggl(x(t)\biggr)^2 + \left(\frac{v(t)}{\omega}\right)^2\quad = \quad\sin^2(\omega t + \phi) + \cos^2(\omega t + \phi)\quad =\quad 1.
\end{equation}
By appropriately choosing the frequency of the hexapole and the length of time spent within, it is possible to prepare the ensemble for a revival of the initial distribution after some additional free propagation afterwards, see the third and fourth rows of Tab.~\ref{phasespacetable}.

\newcommand{\littlefig}[1]{\vspace{-3mm}\newline\includegraphics[clip,trim=1.5cm 2cm 1.2cm 1.7cm,width=3cm]{#1}\vspace{-1mm}}
\renewcommand{\arraystretch}{1}
\begin{table}[t!]
\centering
\caption[Phase Space Manipulations]{
Phase space manipulations are described and visualized. 
Velocity is the vertical axis, Position the horizontal. 
Gray shows the ensemble before the action, black afterwards. 
The first four actions are applied subsequently, the last two are applied directly after initialization.
\label{phasespacetable}}
\begin{tabular}{ C{5cm} | C{5.5cm} | C{4.5cm} }
Action & Effect & Visualization \\
\hline\hline
Initialization 			& Some approximately Gaussian Distribution 	& \littlefig{PSM1.png} 	 \\
\hline
Free Propagation 		& Upper Region to Right, Lower to Left		& \littlefig{PSM2.png} 	 \\
\hline
Hexapole Focusing 		& Clockwise Rotation					& \littlefig{PSM3.png} 	 \\
\hline
Additional Free Propagation	 	& Spatial Density Revival 					& \littlefig{PSM4.png} 	 \\
\hline
Free + Hex + Long Free   	& Velocity Compression 					& \littlefig{PSM5.png} 	 \\
\hline
Free + Hex + Short Free 	& Spatial Compression 					& \littlefig{PSM6.png} 	 \\
\hline
\end{tabular}
\end{table}

Furthermore, it is possible to engineer the applications of hexapole rotation and free flights to achieve compression along one axis, accompanied by density preserving dilation in the opposite axis.
These results parallel the magnification or demagnification of an image achieved by a lens between an image and object plane depending on whether the lens is positioned closer to the image or the object plane.
One key distinction however between the behavior of an optical lens and that of a hexapolar lens is that the latter is almost never operated in the short-lens regime, which has a significant impact on how one thinks about its behavior.
Molecules are often rapidly transiting a hexapole in the longitudinal direction, which means that the device needs to be long in order to apply the desired rotation of the transverse phase space.
By carefully following an ABCD matrix approach, instead of a simple lens-maker equation for the locations of image and object planes, we obtain~\citep[Eq.~9]{Kirste2013}:
\begin{equation}
L_3 = \frac{L_1+\frac{1}{\kappa}\tan(\kappa L_2)}{L_1\kappa\tan(\kappa L_2)-1},
\end{equation}
where $\kappa=\omega/v$ is an inverse length unit directly related to the hexapole strength and inversely to the beam speed, and the $L_i$ are as described in Fig.~\ref{hexdefinitions.png}.
In the limit of small $L_2$, this equation can be reduced to something more familiar.
First, using $\tan(x)\sim x$,
\begin{equation}
L_3 = \frac{L_1+L_2}{L_1L_2\kappa^2-1}= \frac{L_1}{L_1L_2\kappa^2-1},
\end{equation}
and now inverting,
\begin{equation}
\frac{1}{L_3} = L_2\kappa^2 - \frac{1}{L_1},
\end{equation}
so that for $f_H=1/L_2\kappa^2$ we recover the lensmaker formula:
\begin{equation}
\frac{1}{L_1}+\frac{1}{L_3}=\frac{1}{f_H}.
\end{equation}
This focal length has the dependences we would expect, since it reduces (i.e. the lens strengthens) with larger $L_2$ or with larger $\omega$.

\figdave{hexdefinitions.png}{Hexapole Focusing Variables}{Sample trajectories of molecules focused by a hexapole of length $L_2$ after traveling $L_1$ to arrive and an additional $L_3$ afterwards.}{12cm}

This compression of velocity and dilation of position or vice versa is what is known as phase space matching, and is in principle useful for optimizing the transfer of molecules in an ensemble between a source and a device, or between devices.
To flesh this out a bit further, all the devices worked with in the laboratory can be characterized by a phase space acceptance, in principal a six dimensional region but in practice only ever considered two dimensions at a time.
The phase space acceptance typical of the Stark decelerator for example is indicated in Fig.~\ref{phasespaceexamp}, and discussed further in Chap.~\ref{chapter:slowing}.
Unfortunately the phase space generated by a supersonic expansion is rather difficult to ascertain, as discussed previously in Chapt.~\ref{chapter:sourcing}.
One thing which can be said for certain however is that the initial transverse velocity width generated by the supersonic expansion is much larger relative to anything else in the experiment.
This follows from simple intuition as the supersonic expansion only generates a mildly directed beam, which therefore expands significantly transversely away from the valve.
Projected onto transverse phase space, such an ensemble should have a variation of transverse velocity that is on the same order as the forward velocity of the beam, i.e. hundreds of m/s, compared with tens of m/s for the typical depths of devices made with laboratory electric fields.
Nonetheless, the valve may still generate a distribution that is quite narrow in its initial transverse spatial spread.
If the initial distribution is narrower spatially than the $2$~mm of the decelerator, there will be opportunities for an improved loading of the decelerator with appropriate magnification of the device designed to compress velocity and dilate position as in row 5 of Tab.~\ref{phasespacetable}.

\figdave{phasespaceexamp}{Transverse Phase Space Acceptance Examples}{Transverse phase space acceptance of a Stark decelerator operated in various modes. Each mode has a different characteristic ratio of velocity width to spatial width. Some colors are hard to perceive, but all have a width close to $2$~mm and varying heights between $10$ and $30$~m/s.}{10cm}

\section{History}

In the first generation of the experiment, a $5$~cm hexapole was employed and successfully enhanced the molecule number just afterwards by as much as a factor of $16$~\citep[Fig.~6]{Bochinski2004}.
What is less clear however is whether this enhancement really exceeds what could have been had by simply mounting the decelerator closer to the valve and skipping the hexapole entirely.
By the time Hao and I were running the second generation of the experiment~\cite{Sawyer2007}, the hexapole was only providing a $17\%$ effect, as shown in Fig.~\ref{nohexeffect.png}.
I can find no earlier record of the performance of this hexapole, so it is difficult to say whether this observation of no effect with the voltage on/off stems from some HV issue that developed later, from transition between Jordan and PZT valves, or an alignment issue or some other effect.

\figdave{nohexeffect.png}{No Hexapole Effect in Generation Two}{Three traces showing OH population after bunching with the hexapole set at different voltages. A scant $17\%$ effect is observed.}{14cm}

If it is indeed the case that the hexapole in the second generation decelerator never had a significant effect, there are a few possible explanations.
One relates to the influence of skimmer clogging, which can alter the trajectories of molecules so that they appear to have originated from the skimmer tip.
The skimmer tip is too close to the hexapole relative to its focal length for it to have much influence, hence the observation of little effect.
Another possibility is that the initial distribution of molecules in transverse phase space is simply much larger than the phase space acceptance of the decelerator, so that even with no hexapole and pure free propagation, the initial distribution still thoroughly overlaps the acceptance of the decelerator.
This explanation would even jive with the observed influence of the hexapole in~\citep[Fig.~6]{Bochinski2004}, since that observation was made with a detector directly after the hexapole.
This detector could have observed a legitimate hexapole induced spatial density enhancement, but entirely composed of molecules with transverse velocities above the threshold for decelerator acceptance.


\section{Cryogenic Hexapole 1}

Against this dubious backdrop, we decided in the summer of 2018 to begin work on a new hexapole for use with the third generation decelerator.
Our motivations were quite different however from those that motivated the poorly performing hexapole of the second generation.
Instead of seeking to optimize coupling between the valve and decelerator, we were seeking to intentionally distance the two for the sake of reduced clogging potential, without sacrificing the gains in phase space density we had obtained by cryogenically cooling the skimmer.
This was largely motivated by observation of significant decelerator clogging, as described previously in Chap.~\ref{chapter:segregation}.
The new hexapole would this require cooling to cryogenic temperatures, so as to avoid the clogging observed in the decelerator.
In a sense, one can think of the cryogenic hexapole as comparable to cryogenically cooling the decelerator, only much more practically attainable.

\subsection{Hexapole Fundamentals}

A hexapole consists of six alternately charged poles, which generates close to the axis an electric field with the functional form:
\begin{equation}
\left|\vec{E}(r)\right| = 3V_0\left(\frac{r^2}{r_0^3}\right),
\end{equation}
where $V_0$ is the magnitude of the voltage to which each rod is charged, with alternating sign, and $r_0$ is the smallest radius which reaches from the axis to the surface of a pin.
For a molecule with a linear response to the electric field, this quadratic dependence of the electric field magnitude results in a harmonic potential, with the exception of the relatively small region close to the axis where the field is too small to polarize the molecules.
This exception causes a perturbation to the ideal focusing behavior, but is not very significant for OH~\citep[Fig.~2b]{Bochinski2004}, although more so for other species~\citep[Fig.~1]{Kirste2013}.
Specifically, with $V_0=13$~kV and $r_0=5$~mm, which lead to the typical field strengths characteristic of our traps and decelerators, electric fields up to $100$~kV/cm are generated, compared to $3$~kV/cm for polarizing the radical.

\subsection{Core Electrode Design}

My first attempt at a cryogenic hexapole design leveraged some rather complex technologies.
Fearing the possibility of gas mediated discharge of the hexapole electrodes due to the high densities of molecules shortly after the skimmer, I planned to have the electrodes encased in sapphire tubes, so as to avoid the availability of conduction electrons at their surfaces.
These sapphire tubes would in turn sit within a copper block, for maximal thermal conductivity, mechanical alignment, and to provide additional adsorbing surfaces for the Neon.
The fields generated in the planned configuration at its intended voltages are shown in Fig.~\ref{hex1fields.png}.
In panel (a), note how the fields inside the sapphire are particularly small, a byproduct of its large dielectric constant close to $11$.
In panel (b), the electric fields along two axes are shown, one through the electrodes and the other between them.
A quadratic guiding curve is also shown, which corresponds to an angular trapping frequency for OH radicals of $\omega=20.6$~kHz.

\figdave{hex1fields.png}{Electric Fields in Hexapole 1}{(a) Electric field magnitudes and geometric size of the first generation hexapole, composed of 1/16" steel electrodes mounted within 1/8" sapphire tubes mounted within a 1" copper block. Note how the fields inside the sapphire are particularly small, a byproduct of its large dielectric constant close to $11$. (b) Field Strength (also kV/cm) verse distance off axis along various axes as described in the main text.}{\linewidth}

This all of course requires patterning of a honeycomb-like structure on the inside of a long copper block, something which may be readily achieved using a technique known as wire electric discharge machining, wherein a thin sacrificial wire at high voltage is used like a bandsaw blade, and is slowly moved through another metal.
Electric discharges between the wire and the metal being cut cause material to be removed, and the wire is constantly fed so as to avoid its disintegration.
This was achieved according to our specifications without any hitches by the CIRES shop, a shop located at the time of this writing in the chemistry department at CU.

\subsection{Thermal Design}

Some of the thermal considerations pertinent to the design are illustrated in Fig.~\ref{hex1heatloads.png}.
The final mechanical design differs slightly from what is shown in this figure, but the thermal considerations remain the same.
The large copper block within which the hexapole sits has a skimmer affixed to the front and an aperture to the back, and a large clamp-like structure conveys sufficient thermal conductivity without requiring Indium solder.
Confirming this required careful consideration of the force-dependent thermal contact between the clamp and the hexapole, and also between the braid and the clamp.
Thermal conductivity across gold-gold and copper-copper pressure interfaces have been measured~\citep[Fig.~2.7]{Ekin2006}, and exceed grease joints above a certain pressure, see the discussion in~\citep[Sec.~2.6.4]{Ekin2006}.
As can be observed below in Fig.~\ref{hex1view.png}, we settle for a gold-copper joint, which naively would fall somewhere between the gold-gold and copper-copper performances.
Of course nothing is certain, see for example in~\citep[Fig.~5.8]{Hartwig2013} how mixing copper particles into epoxy can actually worsen its thermal conductivity in some temperature ranges.
It is also important to figure out how tight the clamp should be made, so as to benefit from the pressure-dependent thermal conductivity, without irreversibly deforming the copper parts.
I studied this using finite element analysis via Solidworks Premium Simulation package, see Fig.~\ref{hex1force}.
One interesting point worth mentioning is that the geometry of the device leads to a factor of $2\pi$ hoop stress amplification, where the force between the clamp and the cylinder is actually $2\pi$ times bigger than the force applied to the bolts.
This can be derived by conceptually dividing the geometry in half through the bolts and analyzing the static free body diagrams of each component, a standard exercise in statics.
The conclusion is that $1000$~kg on each of two bolts would give $240$~W/\,$^\circ$K for gold-gold and $3$~W/\,$^\circ$K for copper-copper, found by taking the $10\,^\circ$K $50$~kg data in~\citep[Fig.~5.8]{Hartwig2013} and multiplying up to $12000$~kg including the hoop stress amplification.
Of course, this brings to mind the question of validity of the linear scaling of conductivity all the way up to such large forces.
In~\cite{Berman1956}, extremely high loads were investigated, and although their cryogenic hydraulic ram apparatus failed catastrophically during the investigation, they were able to confirm that conductivity increased $5$0-fold for forces increasing $20$-fold up to $4700$~lb, a superlinear scaling to very high forces.

\figdave{hex1force}{Clamp Stress and Strain in Hexapole 1}{
(a) Stress resulting from clamp tightening. 
Annealed copper yields at only $70$~MPa or so, this indicates that $1000$~kg is on the upper end of what should be used on the bolts, and would accompany some yield on the edges of the block.
(b) Strain in the vertical direction only, so as to view what happens to the clamped hexapole mounting copper rod. 
Only $3~\mu$m radial compression is observed, small compared to the $3$~mm diameter sapphire tubes inserted into the rod.
}{\linewidth}

The emissivities guessed for the surfaces shown in panel (b) of Fig.~\ref{hex1heatloads.png} turned out to be optimistic, perhaps because they were approximate room temperature emissivities and not cryogenic, or maybe due to adsorbed gas layers further increasing emissivities.
A key challenge of the design pertains to how voltage is to be fed into the electrodes located inside the sapphire tubes.
A few iterations were made on the design before this was achieved with acceptable surface path-length reduction and suitable mechanical rigidity.
The HV distribution rings visible in Fig.~\ref{hex1heatloads.png} ended up having too much flexibility on their own, so plastic Ultem cuffs were glued directly onto the Sapphire rods, simultaneously increasing surface path length for arcs and adding the required rigidity.

\figdave{hex1heatloads.png}{Heat Conduction and Radiative Loads for Cryo-Hex 1}{(a) Conduction of thermal joints in W/\,$^\circ$K. In the final design, two braids were used, for a total conductivity closer likely above 1 W/\,$^\circ$K. (b) Radiative heat loads on exposed surfaces in mW.}{\linewidth}

An additional consideration for the hexapole is the matter of its alignment with the beamline.
There are two primary strategies by which satisfactory alignment may be obtained:
\begin{enumerate}
\item Contrive a way for alignment to be tweaked from outside of the vacuum chamber, so that final alignment may be guided by actual system figures of merit such as molecule yield after the hexapole. \label{hexitem1}
\item Provide in-vacuum fine-adjust knobs only, and engineer a protocol by which alignment may be obtained before pump-down, and not subsequently lost during either pump-down or cool-down. \label{hexitem2}
\end{enumerate}
Strategy~(\ref{hexitem1}) was employed for our cryogenic skimming work as described previously in Chap.~\ref{chapter:segregation}, but was deemed insufficient for this hexapole work, primarily due to unavailability of the requisite degrees of freedom.
Unlike the skimmer, which only really requires two translational degrees of freedom in order to achieve beam alignment, the hexapole has four alignment parameters, and potentially five depending on whether it is desirable to translate the device along the beam.
There seemed no straightforward way to achieve this across the vacuum envelope, especially given how far above the hexapole the tuning bellows sits.
Thus strategy~(\ref{hexitem2}) was pursued, and the problem of cool-down induced misalignment addressed via a flexible heat linkage\footnote{\href{https://www.techapps.com/}{Technology Applications Inc,} braid P6-502. Tyler Link, at the time of this writing, is always ready to send long emails about his products, at any hour of the day. Contact him directly for a product manual.}\!\!.

\figdave{hex1view.png}{Assembled and Section Views of Hexapole 1}{
(a) A view of Hexapole 1 after assembly. 
(i) 1/8" x 7/8" Ultem rods to insulate but support the structure. 
(ii) Flexible heat linkage, $1.4$~W. 
(iii) Springs enabling fine positioning, four in total. 
(iv) Nichrome heater coil. 
(v) Blurry HV distribution ring barely visible. 
(vi) Invar plate spacers behind steel 1/4-28 nuts. 
(vii) Custom Skimmer, $3$~mm opening. 
(viii) Teflon kinematic cylinder visible for translating but allowing twist. 
(ix) Fine adjust knob, 3/16-100, mounted in a phosphor bronze bushing, torr-sealed into a steel block.
(b) A mechanical section view focusing on the scientific portion of Hexapole 1.
(i) A rear skimmer, one piece with an outer rim that allows grease-only installation.
(ii) Drill holes for slip-fitting thermally insulating Ultem mounts.
(iii) Recess drilled for increasing surface path between the open end of the sapphire rod and the mounting block.
(iv) Plastic  caps glued onto the rods for further voltage safety.
(v) One of six HV holes, \diameter1/2", used for feeding voltage to the electrodes.
(vi) Plastic stubs for increasing surface arc distance in the vicinity of the feedthroughs.
(vii) HV Pin which pokes down through the stubs and contacts electrodes.
(viii) Distribution rings which HV pins thread into.
(ix) Curved contact line where plastic stubs were glued to sapphire tubes.
(x) Recessed solder joint for cryogenic skimmer.

}{\linewidth}

The principal challenge of utilizing a flexible heat linkage is finding a different way to achieve mechanical rigidity without violating thermal isolation.
For this purpose, I selected Ultem (or PEI, polyetherimide), a dimensionally stable engineering thermoplastic with poor thermal conduction and good mechanical properties.
Ultem rods 1/8" wide and 7/8" long can be seen in Fig.~\ref{hex1view.png}(i), which also illustrates other pertinent details of the mounting assembly.
Many polymers feature admirably low thermal conductivity, and the subject of polymer thermal conductivity is treated more fully in~\cite{Hartwig2013}, Chap.~5.
The same selection was once made for cryostats at the RHIC~\cite{Sondericker1994}.
In order to make estimations of the thermal conductivity for long, thin cylinders made of unfilled Ultem 1000 connected between room temperature and the cold apparatus, it is necessary to properly treat the thermal gradient which will necessarily build up within the material.
I address this first in a very general sense.
Suppose a material has a thermal conductivity $c(T)$ which varies with temperature $T$ and is connected as a cylindrical rod between reservoirs at temperatures $T_L$ and $T_H$.
The total heat transfer rate, $\dot{Q}$, must be the same through every slice of the rod, neglecting radiative heat loads, and will induce a temperature change through a thin slice $dx$ of the rod of the amount
\begin{equation}
dT = \frac{\dot{Q}}{c(T)\cdot A / dx},
\end{equation}
where $A$ is the cross sectional area of the rod. Rearranging and integrating:
\begin{equation}
\int\limits_{T_L}^{T_H}c(T)dT=\int\limits_0^L\frac{\dot{Q}dx}{A},
\end{equation}
\begin{equation}
\bar{c}\,\bigg|_{T_L}^{T_H} = \dot{Q}L/A,\label{qdoteq}
\end{equation}
where $\bar{c}$ is the antiderivative of $c$.
We may proceed using the data from~\cite{Sondericker1994} for slightly glass-filled Ultem 2100, which shows $0.04$~W/m$^\circ$K at $10\!~^\circ$K, $0.4$~W/m$^\circ$K at $300\!~^\circ$K, and an approximately straight line on a log-log plot in between.
It follows that $c(T)$ may be approximated as:
\begin{equation}
c(T) = \alpha T^\beta,\quad \beta = \frac{\log\left(\frac{0.4}{0.04}\right)}{\log\left(\frac{300}{10}\right)}=0.68,\quad \alpha = \frac{0.4}{300^{0.68}}=8.5\times10^{-3}\text{ W/m/}(^\circ\text{K})^{\beta+1},
\end{equation}
so that,
\begin{equation}
\bar{c}\,\bigg|_{T_L}^{T_H} = \frac{\alpha}{\beta+1} T^{\beta+1}\,\bigg|_{10^\circ\text{K}}^{300^\circ\text{K}} = 73\text{ W/m}\label{eq73},
\end{equation}
and finally with $A=7.9$~mm$^2$ and $L=20$~mm, we have rearranging Eq.~\ref{qdoteq}:
\begin{equation}
\dot{Q} = 73\text{ W/m}\times7.9\times10^{-6}\text{ m}^2/ (0.02\text{ m})=0.029\text{ W}.
\end{equation}
A table of thermal conductivity integrals with $T_L=4\,^\circ$K and various $T_H$ is provided in~\citep[App.~A2.1]{Ekin2006}. No value is included for Ultem, but the listed polymers have very similar values to that found above in Eq.~\ref{eq73}, for example Teflon is listed as $70$~W/m.
It is also worth pointing out that the conductivity integral between room temperature and $4\,^\circ$K only differs from that between room temperature and $50\,^\circ$K by $10\%$. 
This is symptomatic of the asymmetric distribution of thermal gradient, with most of it occurring in a very small section, and the majority of the plastic closer to room temperature since this is where the thermal conductivity is larger.

\subsection{Performance}

The hexapole did not even come close to performing as intended, particularly as far as high voltage is concerned, with arcing behavior observed at voltages only one third of targets.
One of the first datasets is shown in Fig.~\ref{hexinverteffect.png}.
A surprising effect was observed, whereby the signal was first decreased and then increased.
At first it was suspected that this could relate to the focusing behavior of the device in some way, since by analogy with lenses over-focusing can be problematic.
This does not make sense at such low voltages however, and does not explain the even more surprising observation of hysteretic behavior, where the signal with the hexapole off was observed to decrease dramatically after having operated the hexapole at voltage for a while.

\figdave{hexinverteffect.png}{OH After the Hexapole}{
Flight profiles of OH after the Hexapole. Voltage is seen to have an effect. 
The effect is sometimes disordered, and shows strong hysteresis. 
Note how for the $|e\rangle$ state, signal decreases monotonically with voltage, as one might expect, but for the $|f\rangle$ state, signal first decreases before increasing. 
Upon returning to the $0$~kV trace, the value is larger than both the original $0$~kV and $2$~kV traces. 
These traces were recorded chronologically in the same order as the legend.
The data were collected on November 27, 2018, with a $16.5$~cm valve position setting, which in this setup corresponded to a $10$~cm true valve skimmer distance, a $9.7~^\circ$K hexapole temperature, and an ND1 filter on the fluorescence collection stack.
}{\linewidth}

Ultimately, these effects were attributed to the buildup of very significant patch charges on the surface of the sapphire, significant enough to simulate the hexapole having been charged to several kilovolts of voltage.
This effect was in fact anticipated during the design stage, and a workaround developed- regular polarity reversal.
In the design stage, it was expected that charges might buildup due to ionization of carrier gas upon interaction with the electric field and the cryogenic sapphire surface, but in fact a more potent mechanism exists, see Fig.~\ref{surfacemech.png}.
Despite fillets designed for minimizing surface currents, conduction electrons from within the grounded copper block are likely able to migrate out onto the surface of the sapphire tubes.
This also suggests that charging the hexapole with only negative voltage and ground could help reduce the effect, since the copper block only has negative charges to donate to the surface, not positive.

\figdave{surfacemech.png}{Hexapole Surface Arcing Mechanism}{
This schematic diagram illustrates how surface charges donated from the conductive grounded hexapole mounting block (blue) are able to rather efficiently coat the surface of the sapphire tubes insulating the positively charged hexapole electrodes (black). }{8cm}

While true voltage alternation requires complex tripolar switch output configuration, effective alternation may be achieved with a unipolar setup by connecting each set of three hexapole electrodes to the output of a switch which connects between ground and minus high voltage.
The switches can then take turns charging the hexapole, so that the field direction reverses each time, rod polarity is always negative relative to the grounded copper block, and unipolar push-pull architecture is maintained.
Operating in this manner did indeed resolve patch charge issues, and led to the data shown in Fig.~\ref{hexinvertsolve.png}.
One downside of this mode of operation is the reduced maximum hexapole voltage which may be applied, assuming that availability of MOSFET switches limits the maximum absolute value of voltage which may be applied.
There is also the important question of what influence the non-bipolar hexapole voltages have on fringing field distributions.
Generally speaking, the fringe field magnitude should become much larger, since now on average the hexapole rods are charged instead of neutral.
To some extent, this may be a welcome effect as far as maintaining polarization is concerned, since without any electric field, the well focused \fpm3 molecules would be liable to re-project into \fpm1 after spending time in a region of little or no field.


\figdave{hexinvertsolve.png}{OH post-Hexapole, Patch Charges Resolved}{
Peak OH population, various valve-skimmer distances, several voltages.
Much longer averaging and runtimes without hysteresis and with sensible trace ordering.
Vertical axis is arbitrary, the numbers may be derived from an area integrating script applied to raw data traces akin to those in Fig.~\ref{hexinverteffect.png}.
}{\linewidth}

As far as the thermal performance of the device, a thermal conduction of $1.4$~W/$^\circ$K was achieved, based on the observed second stage cryostat temperature of $7.2~^\circ$K, which suggests via the capacity of the device that a $3$~W heat load is incurred.
This, together with the measured temperature of $9.4~^\circ$K close to the hexapole suggests that $3$~W leads to $2.2~^\circ$K temperature rise, hence the stated $1.4$~W/$^\circ$K.
While the thermal conductivity is as intended, the heat load is much higher than predicted.
This prompted a thorough investigation that ultimately implicated blackbody radiation as the culprit.
At one point we even disconnected the hexapole from its mounting structure so as to take away any heat load due to the thermally insulating mechanical Ultem mounts.
These were completely exonerated, as the observed temperature after their removal did not detectably improve.

\section{Cryogenic Hexapole 2}

Primarily motivated by the failure of the first Hexapole to hold the desired voltages, we undertook a new hexapole with a completely different high voltage strategy.
Fearing significant delays to our long term goals, I pushed for significant reuse of our earlier design, especially with regard to the mounting structure.
The result is shown in Fig.~\ref{hex2cutview}.
The mounting structure and thermal clamp are completely reused, but the copper hexapole mounting block is replaced with a thin copper sheath for transferring heat to the skimmer, inside of which sits an Ultem plastic mounting tube, and inside of that a more traditional stainless steel hexapole.
The steel is allowed to reach some equilibrium temperature set by the balance between weak heat transfer through the Ultem tube to the cryogenic copper transfer tube, and radiative coupling of the high emissivity Ultem to the room temperature vacuum chamber.
A room temperature skimmer is used as a back aperture, and HV distribution is simplified and achieved via steel setscrews that also fix the position of the internal hexapole.
These screws sit in a plane orthogonal to the section plane and are thus not visible.

\figdave{hex2cutview}{Section View of Hexapole 2}{
This section view shows Hexapole 2, and how it was designed to fit inside of the same mounting structure used for Hexapole 1.
(i) A copper transfer sheath transfers heat from the skimmer.
(ii) An insulating Ultem tube extends the entire length of the hexapole, allowing a stainless steel hexapole to be voltage isolated within.
(iii) Ultem stalks are used like in Hexapole 1, but this time with threaded PEEK 2-56 screws connecting them directly to the Ultem sheath.
(iv) One of four HV rings mount to every other hexapole rod in two places each. The rearmost rings also connect to HV wires via setscrews mounted in the Ultem orthogonal to the section plane and this invisible here.
(v) One of six hexapole rods, about 50\% longer than in Hexapole 1, chosen so as to increase effective hexapole position tuning flexibility by varying the turn-on and turn-off of the device.
(vi) Commercial rear skimmer, with nearby gas egress openings visible in the Ultem tube.}{14cm}

Cryogenic hexapole 2 was a very quick turnaround relative to hexapole 1, with first tests occurring within one month of those for hexapole 1.
The second hexapole performed significantly better as far as voltage application, by more than a factor of two.
However it seemed to be clogging limited, showing an earlier falloff with respect to reducing valve-skimmer distance, and a weaker peak signal compared with hexapole 1.
These results are summarized in Fig.~\ref{hex12comp}.
One obvious explanation for this would be the increased temperature of the hexapole, a sacrifice made in light of the importance of first verifying voltage capability before anything else.
It is unsettling however that the first hexapole appears to perform better even at larger distances where the clogging should not yet dominate.
This could be attributed to differences in the alignment of the collection system and laser, but it is difficult to say.

\figdave{hex12comp}{Comparing Hexapoles 1 \& 2}{
Hexapoles 1 and 2 are compared regarding maximum signal with no applied voltage as a function of valve-skimmer distance. 
The second hexapole turns over at larger distance and appears slightly lower in the large distance wing relative to the first.
The original figure was generated on January 7th, 2019.
}{\linewidth}

\subsection{Cryogenic Hexapole 2.5}

Hoping to obtain the best of both worlds, an updated version of Hexapole 2 was created capable of reaching full cryogenic temperatures while leveraging the favorable voltage performance of Hexapole 2.
A section view showing this is available in Fig.~\ref{hex2p5cutview}.
Relative to Hexapole 2, Hexapole 2.5 features HV safety shoulders built together with the HV tube as a single piece, so that a cryogenic copper shield can slide over the entire Ultem HV tube instead of only covering a small portion and serving as an interconnect to the cryogenic skimmer.
To our chagrin, the device featured a significantly worsened high voltage performance relative to Hexapole 2, which ultimately turned out to be a fault related to surface arcing on the Ultem tube itself, despite this not having limited behavior in Hexapole 2.
This was confirmed by successively disassembling the device, pumping back down, and applying high voltage.
Even with all of the shielding and even the mounting structure removed, leaving only the Ultem mount and the steel hexapole within, the arcing persisted.

The reason for the poor performance of the Ultem remains a mystery.
Organic polymers are not inherently bad, as evidenced by the ubiquitous use of polymethacrylate, or acrylic, in HV situations~\cite{Miller1989}.
In fact, organic polymers often outperform glasses and ceramics, possibly due to their reduced dielectric constant, which controls field amplifications near voids and triple points~\citep[pp.~776]{Miller1989}.
Another possible explanation of the difference in performance between Hexapoles 2 and 2.5 relates to their surface treatment, with the latter much more smooth than the former.
Evidence exists for a benefit to surface roughness~\citep[Sec.~8.3.11]{Miller1995}.
It is also possible that cryogenic temperatures negatively influence performance, as observed for some polymers~\citep[Sec.~14.4]{Mazurek2007}.

\figdave{hex2p5cutview}{Section View of Hexapole 2}{
This section view shows Hexapole 2.5.
(i) The outer copper sheath extends past the thermal block, all the way to the rear of the hexapole, and the Ultem tube inside is 1/16" thick everywhere, rather than 1/8" for half of its length in Hexapole 2, see Fig.~\ref{hex2cutview}.
(ii) The rear skimmer is machined together with a cylindrical can designed to mate to the outer diameter of the outer copper sheath, like the rear aperture of Hexapole 1 except in that design the mate was on the ID. Holes outside this aperture skimmer allow steady state gas egress.
(iii) A large box-like first-stage radiation shield is installed.
(iv) An Ultem part directly mounts the radiation shield to the second stage thermal clamp, but with extremely low thermal conduction thanks to a 3D printed hole array.
(v) A smaller shield boxes covers the front of the hexapole, except for the skimmer cone. This screws into the larger shield box.
(vi) Another shield box covers the rear, except for an egress hole for the beam. These shield parts are all bent to shape and then annealed.
(vii) Macor bushings insulate \diameter 1/2" holes for HV feedthroughs. The other sits inside the smaller rear shield box in the unshown sector.
}{14cm}

As far as cryogenics are concerned, hexapole 2.5 performed better than any other, validating the blackbody shielding technique incorporated.
Surprisingly, this did not lead to significant clogging mitigation relative to hexapole 2 at zero electric field where the high voltage performance is not relevant.
In fact, by varying temperatures, the magnitude of the cryogenic benefit was directly investigated, and found to be far smaller than anticipated, see Fig.~\ref{hex25clogging}.
One explanation for this could be the negative influence of surfaces positioned orthogonally to the beamline, such as those formed by the HV distribution rings in Hexapoles 2 and 2.5 but not on Hexapole 1.
Another explanation could relate to the adsorptive properties of the materials involved in the two hexapoles.
In Hexapole 1, the surfaces presented to the molecules were lightly greased sapphire and semi-rough copper from the wire-EDM process, while in Hexapole 2 and 2.5 the surfaces are ungreased highly polished steel.
It is conceivable that the former would perform better, in which case one simple test would be to lightly grease the stainless steel.


\figdave{hex25clogging}{Clogging Behavior of Hexapole 2.5}{
Time of flight traces for different valve-skimmer distances and at two temperatures elucidate the clogging performance of Hexapole 2.5.
}{\linewidth}

Some of the technologies involved in the cryogenic performance of Hexapole 2.5 are worth some further discussion.
Firstly, the plastic component visible in Fig.~\ref{hex2p5cutview}(iv) demonstrates the power of 3D printable thermoplastics for custom thermally insulating structures with good mechanical rigidity.
Thermal simulations of this part indicate conductance in the vicinity of $10\,^\circ$K/mW, which includes a factor of $2.5$ achieved by the convenient removal of material in a hole pattern~\ref{hex25thermiso}
This patterning not only reduces thermal conductivity, but also aids in the removal of 3D printing scaffold material and reduces print time.
It is also possible to reduce the thermal conductivity of 3D printed parts by reducing infill, so that voids are intentionally left inside of a structure.
This seems inadvisable given the likelihood of virtual leaks to the vacuum chamber, although this may not be relevant for parts that will sit between two cryostat stages and therefore lie entirely below the freezing point of all relevant gases after cool-down.
Secondly, the radiation shields are manufactured using sheet-metal bending techniques.
Sheet-metal bending techniques are not often utilized in the laboratory, often reserved for high throughput manufacturing, but in Hexapole 2.5 they really enable geometries that would be vastly more challenging to achieve with traditional fastener-based methodologies.
Using appropriate design modules in Solidworks, the radiation shields are designed to be folded into the desired shape after first being water-jetted out of copper sheet.
At the time of this writing, water-jetting is available in the physics shop with nearly next-day service and only 1-2 hour operator costs.
Bending is hard to do accurately, but this is generally not required for radiation shielding, and if accuracy is required, commercial CNC bending is available with 0.005" tolerances on hole positions for the final folded part\footnote{\href{https://rapidmanufacturing.com/rapid-sheet-metal/}{Rapid Sheet Metal} offers 10-day service and something like \$500 per design, way less with quantity.}.
The act of bending reduces thermal conductivity by work-hardening the joint, but where conductivity is critical, this can be addressed by annealling.

\figdave{hex25thermiso}{Hexapole 2.5 Radiation Shielding Isolator}{Thermal simulations of an Ultem isolator used as a mechanical interconnect between Hexapole 2.5 and its radiation shields, with (b) and without (a) extra holes. In each case, a $1$~mW load is applied across the device, and the base is fixed at $0\,^\circ$K, so that the reported temperature on the upper surface of the device rises with smaller conductance.}{\linewidth}



\section{Guided Flux}

One important way to gauge the performance of these hexapoles is to study the flux of molecules which they are able to couple into the decelerator when it is run in a guiding mode.
We can use this to compare the hexapoles together with earlier measurements without a hexapole to get at the key question of whether a cryogenic hexapole is capable of improving performance relative to the noncryogenic, hexapole-free, commercial skimmer approach.
A selection of several measured fluxes in guiding spanning the hexapoles discussed here are shown in Tab.~\ref{hexguidetable}.
In all cases, the skimmer used features a $3$~mm opening diameter, and the DC guiding voltage used is $6$~kV, a value which gives a transverse phase space acceptance comparable to that of F mode decelerator operation, see Sec.~\ref{sec:afd}.
Overall, only Hexapole 1 is seen to outperform the commercial skimmer option, and that by less than a factor of two.
It is still an accomplishment to have exceeded the previous best, but a far cry from the thirty-fold gain comparing a room temperature and cryogenic skimmer without any coupling to the decelerator~\cite{Wu2018}.

\begin{table}[t!]
\newcommand{\hb}{\qquad\qquad\qquad}
\renewcommand{\arraystretch}{1.3}
\centering
\caption[Guiding Coupling Efficiency Comparison]{
The historical record of coupling molecules into the decelerator, with and without hexapoles, is presented. 
Hexapole 1 is seen to provide the greatest peak signal by $48\%$, and the greatest total flux by $75\%$.
\label{hexguidetable}}
\begin{tabular}{ C{2.8cm} | C{3.3cm} | C{2.8cm} | C{2.8cm} | C{2.2cm} }
Date & Situation & Velocity Spread (m/s)& Peak Signal (photons/shot) & Total Flux (arb) \\
\hhline{=|=|=|=|=}
\hb Aug. 2, 2018\hb\mbox 			& No Hexapole, \hb$8^\circ$\!~K Skimmer 	& 10.7 & 3.2 & 95 \\
\hline
\hb Aug. 21, 2018\hb\mbox 		& No Hexapole, \hb Hot Skimmer				& 11.1 & 4.8 & 156\\
\hline
\hb Dec. 04, 2018\hb\mbox 		& Hexapole 1									& 14.1 & 7.1 & 275\\
\hline
\hb Jan. 07, 2019\hb\mbox	 	& Hexapole 2									& 11.7 & 3.3 & 119 \\
\hhline{=|=|=|=|=}
\end{tabular}
\end{table}


%It is also worth pointing out that the flux of molecules which may be detected has a codependency on the guiding and the hexapole voltage, as shown in Fig.~\ref{hexandguide}.
%In other words, for a low guiding voltage, the guided flux saturates at a lower hexapole voltage.
%This constitutes a direct observation of the phase space overfilling effect, and can be seen as a way of predicting the relevance of the hexapole for deceleration modes with more significant transverse phase space acceptance.



\section{Cryogenic Hexapole 3}

After an interlude for attempts at collisional studies in the decelerator and between parallel plates, it was time to move on to other things, but the lessons learned on the first two and a half hexapoles could one day be brought to full fruition in a final iteration of the device.
By abandoning the emphasis in Hexapoles 2 and 2.5 of reusing the mounting structure developed for Hexapole 1, it should be possible to further increase the surface path distance between electrodes, by say an additional factor of 3-5 beyond the factor of 3 already implemented between Hexapoles 1 and 2.
In addition, by carefully following the principles of triple point shielding laid out in~\citep[Sec.~4.3.3]{Faircloth2013}, extra safety should be able to be obtained.
In particular, true shielding of the triple points that exist between the Ultem tube and the internal HV distribution mounts of Hexapoles 2 and 2.5 is not really possible given the external copper tube and requirement of sliding installation inside.

In addition to making a further dramatic increase in surface arcing path-length and triple point shielding, the next generation hexapole will need to address the observation of limited cryoadsorption performance.
In addition to the reduction or removal of orthogonal surfaces and the investigation of surface materials and the performance of grease, another possibility would be to reduce all surfaces, even parallel ones, presented to the beam-line.
One way to move in this direction would be to reduce the radius of the hexapole rods, which would introduce some dodecapolar deviations from the ideal hexapole fields, likely a minor effect on top of the clogging behavior.
Another rather drastic possibility would be to remove the front skimmer entirely.
It is difficult to predict what would result from this, but if the skimmer is still reducing the beam flux relative to what it could be with no skimmer, a big win could be obtained.
Simultaneously, without the skimmer the flux onto the hexapole rods would be significantly increased, exacerbating any clogging issues associated with these surfaces.

An important factor to keep in mind is the effective transverse size of the source coming from the valve.
If the valve generates molecules with a broad enough spread in their initial transverse coordinate relative to what the decelerator can accept, this can pose a limit on what can be obtained by adding a hexapole.
The reason for this is that with a suitably broad initial distribution, even an extended time in free flight may be incurred without significant broadening of the portion of phase space that overlaps with the acceptance of the decelerator.
It is difficult to measure this initial transverse spread of the molecules, but simulations of transverse oscillations in the decelerator suggest the value may be somewhat large, see Sec.~\ref{ttosec}.

\section{Conclusions}

Several cryogenic hexapoles have been designed and tested.
High Voltage issues have proven more pervasive than initially expected.
Clogging issues have proven difficult to fully understand.
As far as the absolute comparison between what may be obtained with a cryogenic skimmer and hexapole and what may be obtained with a traditional skimmer, thus far nearly a factor of two has been demonstrated, despite the high voltage and clogging underperformance.












\ifx\justbeingincluded\undefined
\bibliographystyle{unsrtDR}	% or "siam", or "alpha", etc.
\bibliography{allrefs}		% Bib database in "allrefs.bib"
\end{document}
\fi
