
\chapter{Introduction}
\label{iii}

Cold and Ultracold molecules have long promised a wealth of new opportunity to the AMO physics community, and are now truly blossoming in this regard. Associated ultracold atoms are available at unprecedented phase space densities, as are directly laser-coolable molecules. This in turn imposes some welcome pressure on experiments with species directly cooled and decelerated to finally deliver on promises of high density and in-trap collisions.
%
%\begin{figure}[htbp]
%\caption[Cylinder and measurements]{
%This diagram of a cylinder and various
%measurements and quantities was actually
%made using {\bf xfig}, a freeware
%drawing program for Unix systems.
%Diagrams can be exported directly to PDF
%files, the preferred format for
%vector graphics.  Vector graphics can
%be magnified indefinitely without degradation,
%whereas bitmap images (JPG and PNG)
%must be pretty high-resolution if you don't
%want them looking all pixellated when
%magnified.
%}
%    \begin{center}
%\includegraphics[width=100mm]{figs/cyl.pdf}
%    \end{center}
%\label{xfigDiagram}
%\end{figure}
%
%
%\begin{figure}[htbp]
%    \caption[Bitmap images]{
%The JPEG bitmap format is great for photos but
%crummy for diagrams (including drawings, graphs,
%charts) because it can't gracefully handle sharp edges.
%Note the same bitmap image below from a PNG file and
%from a JPG file; the latter shows characteristic
%``ringing'' at sharp edges -- including text!
%Seriously, magnify and look closely at the JPG's
%awful lines and edges.
%Vector-format PDF is the best for diagrams, but
%if you must use a bitmap image, let it be PNG.
%~ (Left: file {\it drawing.png}.
%Right: file {\it drawing.jpg}.)
%}
%    \begin{center}
%\includegraphics[width=70mm]{figs/drawing.png}
%${}^{}$ ~
%${}^{}$ ~
%${}^{}$
%\includegraphics[width=70mm]{figs/drawing.jpg}
%    \end{center}
%\label{bitmapImages}
%\end{figure}
%


\section{Direct Cooling}

Direct cooling refers to the preparation of molecules of interest at typical temperatures and subsequent additional cooling. This is in contrast to associative type methods which first cool constituents atoms.

\section{Laser Cooling}

Much progress has been made using laser cooling methods directly on molecules.

\section{Association}

%
%\begin{table}[htb]
%    \caption[Example of a table with its own footnotes]{
%	Here is an example of a table with its own footnotes.
%	Don't use the $\backslash${\tt footnote} macro if you
%	don't want the footnotes at the bottom of the page.
%	Also, note that in a thesis the caption goes
%	\emph{above} a table, unlike figures.
%	}
%    \begin{center}
%    \begin{tabular}{||l|c|c|c|c||} \hline
%	& $S$ & $P$ &   $Q^{\ast}$  & $D^{\dagger}$ \\	% footnote symbols!
%	wave form & (kVA) & (kW) & (kVAr) & (kVAd) \\  \hline \hline
%	Fig.  \ref{xfigDiagram}a  & 25.48 & 25.00 & -2.82 & 4.03 \\ \hline
%	Fig.  \ref{xfigDiagram}b  & 25.11 & 18.02 & -9.75 & 14.52 \\ \hline
%	Table \ref{pdftable}  & 24.98 & 22.26 & 9.19 & 6.64 \\ \hline
%	Table \ref{powertable}  & 23.48 & 15.00 & 6.59 & 16.82 \\ \hline
%	Fig.  \ref{pyramid}  & 24.64 & 22.81 & -0.44 & 9.3 \\ \hline
%	\end{tabular}
%   \\ \rule{0mm}{5mm}
%   ${}^\ast$kVAr means reactive power.		% footnote symbol
%\\ ${}^\dagger$kVAd means distortion power.	% footnote symbol
%\end{center}
%\label{powertable}
%\end{table}


